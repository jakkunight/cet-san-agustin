\documentclass[landscape, a4paper, 10pt]{article}
\usepackage[spanish]{babel}
\usepackage[utf8]{inputenc}
\usepackage{array}
\usepackage{graphicx}
\usepackage{tabularx}
\usepackage{longtable}
\usepackage[a4paper,margin=1.5cm,landscape]{geometry}
\usepackage{marginnote}
\usepackage{rotating}
\usepackage{blindtext}
\usepackage[dvipsnames]{xcolor}
\usepackage{tcolorbox}
\newtcolorbox{bgshadow}{
	arc=0pt,
	boxrule=2pt,
	colback=Apricot,
	width=\textheight,
	halign=center
}
% Tamaño de las celdas:
\newcommand{\smallcellwidth}{0.7in}
\newcommand{\normalcellwidth}{1.2in}
\newcommand{\bigcellwidth}{2.0in}
% Identificación:
\newcommand{\profesor}{Santiago Wu}
\newcommand{\discipline}{Robótica y Programación}
\newcommand{\currentyear}{2023}
\newcommand{\institution}{Centro Educativo y Técnico San Agustín}
\newcommand{\group}{1er Grado}
\newcommand{\CEGSA}{cegsa-logo.png}
\newcommand{\CETSA}{cetsa-logo.png}
% -
\begin{document}
	% Etiqueta:
	%\reversemarginpar\marginnote{
	%	\begin{turn}{90}
	%		\begin{bgshadow}
	%			Centro Educativo y Técnico San Agustín 2023
	%		\end{bgshadow}
	%	\end{turn}
	%}
	% Cabecera:
	\begin{tabularx}{\textwidth}{ >{\raggedright\arraybackslash}X >{\centering\arraybackslash}X >{\raggedleft\arraybackslash}X }
		\includegraphics[width=0.3\textwidth]{\CEGSA} &
		\textbf{PLAN ANUAL DE CLASES} &
		\includegraphics[width=0.3\textwidth]{\CETSA}
	\end{tabularx}
	% Identificación:
	\begin{tabularx}{\textwidth}{ >{\raggedright\arraybackslash}X >{\raggedright\arraybackslash}X >{\raggedright\arraybackslash}X }
		Docente: \profesor &
		Turno: - &
		Año: \currentyear \\
		Disciplina: \discipline &
		Grado/Curso: \group &
		 \\
	\end{tabularx}
	% Contenido:
	\centering
	\begin{longtable}{|m{\smallcellwidth}|p{\normalcellwidth}|p{\bigcellwidth}|p{\bigcellwidth}|p{\normalcellwidth}|p{\normalcellwidth}|p{\normalcellwidth}|}
		\hline
		\textbf{Mes} &
		\textbf{Contenido/Unidad Temática} &
		\textbf{Capacidades} &
		\textbf{Indicadores} &
		\textbf{Recursos Didácticos/Uso de TIC's} &
		\textbf{Instrumentos de Evaluación} &
		\textbf{Proyectos Disciplinarios} \\
		\hline
		\endhead
		% Plan de Febrero:
		Febrero &
		Computadoras. Concepto y elementos constituyentes.\par
		Sistema Operativo. Concepto, ejemplos y funciones.\par
		Periféricos. Concepto, tipos, uso y ejemplos.\par
		Archivos y programas. Concepto, tipos, usos y ejemplos. &
		\begin{itemize}
			\item Utilizar con normalidad una computadora en cualquiera de sus formatos (PC, Móvil, Portátil, etc).
		\end{itemize} &
		\begin{itemize}
			\item Utilizar correctamente los periféricos más comunes (mouse, teclado, monitor, etc).
			\item Utilizar correctamente los archivos y programas más comunes (imágenes, audio, texto plano, etc).
			\item Comprender el concepto y uso de las computadoras y los sistemas operativos.
			\item Ayudar a sus compañeros a comprender los conceptos y/o habilidades aprendidas en clase.
			\item Utilizar diversos medios para obtener datos y/o información (preguntar a personas de confianza, deducciones, experimentación, etc).
			\item Mostrar curiosidad por los elementos de su entorno.
		\end{itemize} &
		Ilustraciones, computadoras, juegos, etc. &
		Preguntas orales, Ejercicios de aplicación, Juegos, etc &
		 - \\
		\hline
		% Plan de Marzo:
		Marzo &
		 &
		\begin{itemize}
			\item Elaborar programas que resuelvan problemas simples presentados en su entorno.
			\item Aplicar las instrucciones de un programa propuesto a la resolución de problemáticas sencillas presentadas en su día a día.
		\end{itemize} &
		\begin{itemize}
			\item 
		\end{itemize} &
		 &
		 &
		 \\
		\hline
		% Plan de Abril:
		Abril &
		 &
		\begin{itemize}
			\item 
		\end{itemize} &
		\begin{itemize}
			\item 
 		\end{itemize} &
 		 &
		 &
		 \\
		\hline
		% Plan de Mayo:
		Mayo &
		 &
		\begin{itemize}
			\item 
		\end{itemize} &
		\begin{itemize}
			\item 
 		\end{itemize} &
		 &
		 &
		 \\
		\hline
		% Plan de Junio:
		Junio &
		 &
		\begin{itemize}
			\item 
		\end{itemize} &
		\begin{itemize}
			\item 
 		\end{itemize} &
		 &
		 &
		 \\
		\hline
		% Plan de Julio:
		Julio &
		 &
		\begin{itemize}
			\item 
		\end{itemize} &
		\begin{itemize}
			\item 
 		\end{itemize} &
		 &
		 &
		 \\
		\hline
		% Plan de Agosto:
		Agosto &
		 &
		\begin{itemize}
			\item 
		\end{itemize} &
		\begin{itemize}
			\item 
 		\end{itemize} &
		 &
		 &
		 \\
		\hline
		% Plan de Septiembre:
		Septiembre &
		 &
		\begin{itemize}
			\item 
		\end{itemize} &
		\begin{itemize}
			\item 
 		\end{itemize} &
		 &
		 &
		 \\
		\hline
		% Plan de Octubre:
		Octubre &
		 &
		\begin{itemize}
			\item 
		\end{itemize} &
		\begin{itemize}
			\item 
 		\end{itemize} &
		 &
		 &
		\\
		\hline
		% Plan de Noviembre:
		Noviembre &
		 &
		\begin{itemize}
			\item 
		\end{itemize} &
		\begin{itemize}
			\item 
 		\end{itemize} &
		 &
		 &
		 \\
		\hline

	\end{longtable}
	\pagebreak[4]
	% ----------------------------------------------------------------------------------------------------------------------------------------
\end{document}
