\documentclass[landscape, a4paper, 10pt]{article}
\usepackage[spanish]{babel}
\usepackage[utf8]{inputenc}
\usepackage{array}
\usepackage{graphicx}
\usepackage{tabularx}
\usepackage{longtable}
\usepackage[a4paper,margin=1.5cm,landscape]{geometry}
\usepackage{marginnote}
\usepackage{rotating}
\usepackage{blindtext}
\usepackage[dvipsnames]{xcolor}
\usepackage{tcolorbox}
\newtcolorbox{bgshadow}{
	arc=0pt,
	boxrule=2pt,
	colback=Apricot,
	width=\textheight,
	halign=center
}
% Tamaño de las celdas:
\newcommand{\smallcellwidth}{0.7in}
\newcommand{\normalcellwidth}{1.2in}
\newcommand{\bigcellwidth}{2.0in}
% Identificación:
\newcommand{\profesor}{Santiago Wu}
\newcommand{\discipline}{Robótica y Programación}
\newcommand{\currentyear}{2023}
\newcommand{\institution}{Centro Educativo y Técnico San Agustín}
\newcommand{\group}{3er Grado}
\newcommand{\CEGSA}{cegsa-logo.png}
\newcommand{\CETSA}{cetsa-logo.png}
% -
\begin{document}
	% Etiqueta:
	%\reversemarginpar\marginnote{
	%	\begin{turn}{90}
	%		\begin{bgshadow}
	%			Centro Educativo y Técnico San Agustín 2023
	%		\end{bgshadow}
	%	\end{turn}
	%}
	% Cabecera:
	\begin{tabularx}{\textwidth}{ >{\raggedright\arraybackslash}X >{\centering\arraybackslash}X >{\raggedleft\arraybackslash}X }
		\includegraphics[width=0.3\textwidth]{\CEGSA} &
		\textbf{PLAN ANUAL DE CLASES} &
		\includegraphics[width=0.3\textwidth]{\CETSA}
	\end{tabularx}
	% Identificación:
	\begin{tabularx}{\textwidth}{ >{\raggedright\arraybackslash}X >{\raggedright\arraybackslash}X >{\raggedright\arraybackslash}X }
		Docente: \profesor &
		Turno: - &
		Año: \currentyear \\
		Disciplina: \discipline &
		Grado/Curso: \group &
		 \\
	\end{tabularx}
	% Contenido:
	\centering
	\begin{longtable}{|m{\smallcellwidth}|p{\normalcellwidth}|p{\bigcellwidth}|p{\bigcellwidth}|p{\normalcellwidth}|p{\normalcellwidth}|p{\normalcellwidth}|}
		\hline
		\textbf{Mes} &
		\textbf{Contenido/Unidad Temática} &
		\textbf{Capacidades} &
		\textbf{Indicadores} &
		\textbf{Recursos Didácticos/Uso de TIC's} &
		\textbf{Instrumentos de Evaluación} &
		\textbf{Proyectos Disciplinarios} \\
		\hline
		\endhead
		% Plan de Febrero:
		Febrero &
		Computadoras. Concepto y elementos constituyentes.\par
		Sistema Operativo. Concepto, ejemplos y funciones.\par
		Periféricos. Concepto, tipos, uso y ejemplos.\par
		Archivos y programas. Concepto, tipos, usos y ejemplos. &
		\begin{itemize}
			\item Utilizar con normalidad una computadora en cualquiera de sus formatos (PC, Móvil, Portátil, etc).
		\end{itemize} &
		\begin{itemize}
			\item Utilizar correctamente los periféricos más comunes (mouse, teclado, monitor, etc).
			\item Utilizar correctamente los archivos y programas más comunes (imágenes, audio, texto plano, etc).
			\item Comprender el concepto y uso de las computadoras y los sistemas operativos.
			\item Ayudar a sus compañeros a comprender los conceptos o habilidades aprendidas en clase.
			\item Utilizar diversos medios para obtener datos o información (preguntar a personas de confianza, deducciones, experimentación, etc).
			\item Mostrar curiosidad por los elementos de su entorno.
		\end{itemize} &
		Ilustraciones, computadoras, juegos, experimentos sencillos, etc. &
		Preguntas orales, Ejercicios de aplicación, Juegos, etc &
		 - \\
		\hline
		% Plan de Marzo:
		Marzo &
		Lógica de la Programación. Secuencias, Condicionales, Bucles, Bucles de Eventos. &
		\begin{itemize}
			\item Comprender el concepto de secuencialidad.
			\item Comprender el concepto de condicionalidad.
			\item Comprender el concepto de repetición.
			\item Comprender el concepto de evento.
			\item Asimilar los principios básicos de la lógica (PI, PRS, PNC, PTE, PC, etc).
			\item Asociar causas a efectos y viceversa.
			\item Respetar las reglas establecidas en diversas situaciones.
		\end{itemize} &
		\begin{itemize}
			\item Jugar respetando las reglas previamente establecidas.
			\item Planear estrategias para ganar juegos.
			\item Ejecutar estrategias propuestas, tanto por sí mismo como por el profesor, para ganar juegos.
			\item Ayudar a sus compañeros a comprender los conceptos o habilidades aprendidas en clase.
			\item Elaborar programas que resuelvan problemas simples presentados en su entorno.
			\item Aplicar las instrucciones de un programa propuesto a la resolución de problemáticas sencillas presentadas en su día a día.
		\end{itemize} &
		Puzzles, Acertijos, Aplicaciones Web, Juegos, Experimentos sencillos, etc &
		Preguntas orales, Ejercicios de aplicación, Juegos, etc. &
		"¡Jugando también aprendo!"\par
		El proyecto consiste en hacer que los chicos que aún no pueden leer o escribir también puedan ejercitar su razonamiento lógico. 
		La mejor manera que se me ocurre para lograrlo es a través de los juegos, pues revelan mucho acerca del alumno, y permiten inculcarle 
		principios de la lógica, como el Principio de Identidad, estructuras de la programación, como secuencias, condicionales y bucles, 
		y el concepto que domina la programción actual: el objeto. \\
		\hline
		% Plan de Abril:
		Abril &
		Lógica de la Programación. Secuencias, Condicionales, Bucles, Bucles de Eventos. &
		\begin{itemize}
			\item Comprender el concepto de secuencialidad.
			\item Comprender el concepto de condicionalidad.
			\item Comprender el concepto de repetición.
			\item Comprender el concepto de evento.
			\item Asimilar los principios básicos de la lógica (PI, PRS, PNC, PTE, PC, etc).
			\item Asociar causas a efectos y viceversa.
			\item Respetar las reglas establecidas en diversas situaciones.
		\end{itemize} &
		\begin{itemize}
			\item Jugar respetando las reglas previamente establecidas.
			\item Planear estrategias para ganar juegos.
			\item Ejecutar estrategias propuestas, tanto por sí mismo como por el profesor, para ganar juegos.
			\item Ayudar a sus compañeros a comprender los conceptos o habilidades aprendidas en clase.
			\item Elaborar programas que resuelvan problemas simples presentados en su entorno.
			\item Aplicar las instrucciones de un programa propuesto a la resolución de problemáticas sencillas presentadas en su día a día.
		\end{itemize} &
		Puzzles, Acertijos, Aplicaciones Web, Juegos, Experimentos sencillos, etc &
		Preguntas orales, Ejercicios de aplicación, Juegos, etc. &
		"¡Jugando también aprendo!"\par
		El proyecto consiste en hacer que los chicos que aún no pueden leer o escribir también puedan ejercitar su razonamiento lógico. 
		La mejor manera que se me ocurre para lograrlo es a través de los juegos, pues revelan mucho acerca del alumno, y permiten inculcarle 
		principios de la lógica, como el Principio de Identidad, estructuras de la programación, como secuencias, condicionales y bucles, 
		y el concepto que domina la programción actual: el objeto. \\
		\hline
		% Plan de Mayo:
		Mayo &
		Lenguajes de Programación. Concepto, Elementos constituyentes, Usos y Ejemplos. &
		\begin{itemize}
			\item Comprender el concepto de lenguaje.
			\item Utilizar el lenguaje correctamente para expresar distintas ideas.
			\item Aplicar el concepto de lenguaje a las situaciones de su entorno que así lo requieran.
			\item Aplicar las estructuras de control básicas (Secuencia, Condición, Bucle, etc) con naturalidad cuando sean requeridas.
			\item Asimilar los principios básicos de la lógica (PI, PRS, PNC, PTE, PC, etc).
			\item Asociar causas a efectos y viceversa.
			\item Respetar las reglas establecidas en diversas situaciones.
		\end{itemize} &
		\begin{itemize}
			\item Explicar situaciones sencillas de su día a día mediante relaciones condición-acción.
			\item Ordenar ideas expuestas mediante un razonamiento lógico.
			\item Asocia condiciones y conclusiones mediante lógicamente.
			\item Ayudar a sus compañeros a comprender los conceptos y/o habilidades aprendidas en clase.
			\item Elaborar programas que resuelvan problemas simples presentados en su entorno.
			\item Aplicar las instrucciones de un programa propuesto a la resolución de problemáticas sencillas presentadas en su día a día.
		\end{itemize} &
		Puzzles, Acertijos, Experimentos sencillos, etc &
		Preguntas orales, Ejercicios de aplicación, Juegos, etc. &
		""\par \\
		\hline
		% Plan de Junio:
		Junio &
		Exámenes. Evaluación. &
		\begin{itemize}
			\item Aplicar las estructuras de control básicas (Secuencia, Condición, Bucle, etc) con naturalidad cuando sean requeridas.
			\item Asimilar los principios básicos de la lógica (PI, PRS, PNC, PTE, PC, etc).
			\item Asociar causas a efectos y viceversa.
			\item Respetar las reglas establecidas en diversas situaciones.
		\end{itemize} &
		\begin{itemize}
			\item Jugar respetando las reglas previamente establecidas.
			\item Planear estrategias para ganar juegos.
			\item Ejecutar estrategias propuestas, tanto por sí mismo como por el profesor, para ganar juegos.
			\item Ayudar a sus compañeros a comprender los conceptos y/o habilidades aprendidas en clase.
			\item Elaborar programas que resuelvan problemas simples presentados en su entorno.
			\item Aplicar las instrucciones de un programa propuesto a la resolución de problemáticas sencillas presentadas en su día a día.
 		\end{itemize} &
		Ejercicios de evaluación, Preguntas acerca de conceptos clave, Sketches. &
		Ejercicios de evaluación, Preguntas acerca de conceptos clave, Sketches. &
		- \\
		\hline
		% Plan de Julio:
		Julio &
		Lógica de la Programación. Aplicación y práctica. &
		\begin{itemize}
			\item Aplicar las estructuras de control básicas (Secuencia, Condición, Bucle, etc) con naturalidad cuando sean requeridas.
			\item Asimilar los principios básicos de la lógica (PI, PRS, PNC, PTE, PC, etc).
			\item Asociar causas a efectos y viceversa.
			\item Respetar las reglas establecidas en diversas situaciones.
		\end{itemize} &
		\begin{itemize}
			\item Jugar respetando las reglas previamente establecidas.
			\item Planear estrategias para ganar juegos.
			\item Ejecutar estrategias propuestas, tanto por sí mismo como por el profesor, para ganar juegos.
			\item Ayudar a sus compañeros a comprender los conceptos y/o habilidades aprendidas en clase.
			\item Elaborar programas que resuelvan problemas simples presentados en su entorno.
			\item Aplicar las instrucciones de un programa propuesto a la resolución de problemáticas sencillas presentadas en su día a día.
 		\end{itemize} &
		Ejercicios de evaluación, Preguntas acerca de conceptos clave, Sketches. &
		Ejercicios de evaluación, Preguntas acerca de conceptos clave, Sketches. &
		- \\
		\hline
		% Plan de Agosto:
		Agosto &
		Lógica de la Programación. Objetos y Clases. Atributos y Métodos. Constructores y Destructores. &
		\begin{itemize}
			\item Comprender el concepto de clase.
			\item Comprender el concepto de objeto.
			\item Utilizar correctamente clases y objetos para representar situaciones problemáticas simples.
			\item Aplicar las estructuras de control básicas (Secuencia, Condición, Bucle, etc) con naturalidad cuando sean requeridas.
			\item Asimilar los principios básicos de la lógica (PI, PRS, PNC, PTE, PC, etc).
			\item Asociar causas a efectos y viceversa.
			\item Respetar las reglas establecidas en diversas situaciones.
		\end{itemize} &
		\begin{itemize}
			\item Asociar características propias de un objeto al mismo.
			\item Diferenciar clases de objetos.
			\item Definir acciones concretas realizadas por o sobre el objeto.
			\item Jugar respetando las reglas previamente establecidas.
			\item Ayudar a sus compañeros a comprender los conceptos y/o habilidades aprendidas en clase.
			\item Elaborar programas que resuelvan problemas simples presentados en su entorno.
			\item Aplicar las instrucciones de un programa propuesto a la resolución de problemáticas sencillas presentadas en su día a día.
		\end{itemize} &
		Ilustraciones, computadoras, juegos, experimentos sencillos, etc. &
		Preguntas orales, Ejercicios de aplicación, Juegos, etc &
		""\par \\
		\hline
		% Plan de Septiembre:
		Septiembre &
		Lógica de la Programación. Ámbitos privados y públicos. &
		\begin{itemize}
			\item Entender el concepto de ámbito privado.
			\item Entender el concepto de ámbito público.
			\item Utilizar correctamente los ámbitos de los atributos y métodos en la creación de clases y programas.
			\item Aplicar las estructuras de control básicas (Secuencia, Condición, Bucle, etc) con naturalidad cuando sean requeridas.
			\item Asimilar los principios básicos de la lógica (PI, PRS, PNC, PTE, PC, etc).
			\item Asociar causas a efectos y viceversa.
			\item Respetar las reglas establecidas en diversas situaciones.
		\end{itemize} &
		\begin{itemize}
			\item Relacionar propiedades y acciones con los objetos correspondientes en la vida real.
			\item Comprender las interacciones entre los diversos objetos y sus consecuencias.
			\item Manejar múltiples tareas mediante el modelo del bucle de eventos.
			\item Jugar respetando las reglas previamente establecidas.
			\item Ayudar a sus compañeros a comprender los conceptos y/o habilidades aprendidas en clase.
			\item Elaborar programas que resuelvan problemas simples presentados en su entorno.
			\item Aplicar las instrucciones de un programa propuesto a la resolución de problemáticas sencillas presentadas en su día a día.
		\end{itemize} &
		Ilustraciones, computadoras, juegos, experimentos sencillos, etc. &
		Preguntas orales, Ejercicios de aplicación, Juegos, etc &
		""\par \\
		\hline
		% Plan de Octubre:
		Octubre &
		Lógica de la Programación. Aplicación y Práctica. &
		\begin{itemize}
			\item Aplicar las estructuras de control básicas (Secuencia, Condición, Bucle, etc) con naturalidad cuando sean requeridas.
			\item Asimilar los principios básicos de la lógica (PI, PRS, PNC, PTE, PC, etc).
			\item Asociar causas a efectos y viceversa.
			\item Respetar las reglas establecidas en diversas situaciones.
		\end{itemize} &
		\begin{itemize}
			\item Explicar situaciones cotidianas de su día a día aplicando el concepto de repetición.
			\item Hallar patrones en secuencias previamente expuestas.
			\item Detectar la condición de repetición de acontecimientos concretos.
			\item Jugar respetando las reglas previamente establecidas.
			\item Ayudar a sus compañeros a comprender los conceptos y/o habilidades aprendidas en clase.
			\item Elaborar programas que resuelvan problemas simples presentados en su entorno.
			\item Aplicar las instrucciones de un programa propuesto a la resolución de problemáticas sencillas presentadas en su día a día.
		\end{itemize} &
		Ejercicios de evaluación, Preguntas acerca de conceptos clave, Sketches. &
		Ejercicios de evaluación, Preguntas acerca de conceptos clave, Sketches. &
		""\par \\
		\hline
		% Plan de Noviembre:
		Noviembre &
		Lógica de la Programación. Aplicación y práctica. &
		\begin{itemize}
			\item Aplicar las estructuras de control básicas (Secuencia, Condición, Bucle, etc) con naturalidad cuando sean requeridas.
			\item Asimilar los principios básicos de la lógica (PI, PRS, PNC, PTE, PC, etc).
			\item Asociar causas a efectos y viceversa.
			\item Respetar las reglas establecidas en diversas situaciones.
		\end{itemize} &
		\begin{itemize}
			\item Jugar respetando las reglas previamente establecidas.
			\item Planear estrategias para ganar juegos.
			\item Ejecutar estrategias propuestas, tanto por sí mismo como por el profesor, para ganar juegos.
			\item Ayudar a sus compañeros a comprender los conceptos y/o habilidades aprendidas en clase.
			\item Elaborar programas que resuelvan problemas simples presentados en su entorno.
			\item Aplicar las instrucciones de un programa propuesto a la resolución de problemáticas sencillas presentadas en su día a día.
 		\end{itemize} &
		Ejercicios de evaluación, Preguntas acerca de conceptos clave, Sketches. &
		Ejercicios de evaluación, Preguntas acerca de conceptos clave, Sketches. &
		- \\
		\hline

	\end{longtable}
	\pagebreak[4]
	% ----------------------------------------------------------------------------------------------------------------------------------------
\end{document}
