\documentclass[landscape, a4paper, 10pt]{article}
\usepackage[spanish]{babel}
\usepackage[utf8]{inputenc}
\usepackage{array}
\usepackage{graphicx}
\usepackage{tabularx}
\usepackage{longtable}
\usepackage[a4paper,margin=1.5cm,landscape]{geometry}
\usepackage{marginnote}
\usepackage{rotating}
\usepackage{blindtext}
\usepackage[dvipsnames]{xcolor}
\usepackage{tcolorbox}
\newtcolorbox{bgshadow}{
	arc=0pt,
	boxrule=2pt,
	colback=Apricot,
	width=\textheight,
	halign=center
}
% Tamaño de las celdas:
\newcommand{\smallcellwidth}{0.7in}
\newcommand{\normalcellwidth}{1.2in}
\newcommand{\bigcellwidth}{2.0in}
% Identificación:
\newcommand{\profesor}{Santiago Wu}
\newcommand{\discipline}{Robótica y Programación}
\newcommand{\currentyear}{2023}
\newcommand{\institution}{Centro Educativo y Técnico San Agustín}
\newcommand{\group}{1er Grado}
\newcommand{\CEGSA}{cegsa-logo.png}
\newcommand{\CETSA}{cetsa-logo.png}
% -
\begin{document}
	% Etiqueta:
	%\reversemarginpar\marginnote{
	%	\begin{turn}{90}
	%		\begin{bgshadow}
	%			Centro Educativo y Técnico San Agustín 2023
	%		\end{bgshadow}
	%	\end{turn}
	%}
	% Cabecera:
	\begin{tabularx}{\textwidth}{ >{\raggedright\arraybackslash}X >{\centering\arraybackslash}X >{\raggedleft\arraybackslash}X }
		\includegraphics[width=0.3\textwidth]{\CEGSA} &
		\textbf{PLAN ANUAL DE CLASES} &
		\includegraphics[width=0.3\textwidth]{\CETSA}
	\end{tabularx}
	% Identificación:
	\begin{tabularx}{\textwidth}{ >{\raggedright\arraybackslash}X >{\raggedright\arraybackslash}X >{\raggedright\arraybackslash}X }
		Docente: \profesor &
		Turno: - &
		Año: \currentyear \\
		Disciplina: \discipline &
		Grado/Curso: \group &
		 \\
	\end{tabularx}
	% Contenido:
	\centering
	\begin{longtable}{|m{\smallcellwidth}|p{\normalcellwidth}|p{\bigcellwidth}|p{\bigcellwidth}|p{\normalcellwidth}|p{\normalcellwidth}|p{\normalcellwidth}|}
		\hline
		\textbf{Mes} &
		\textbf{Contenido/Unidad Temática} &
		\textbf{Capacidades} &
		\textbf{Indicadores} &
		\textbf{Recursos Didácticos/Uso de TIC's} &
		\textbf{Instrumentos de Evaluación} &
		\textbf{Proyectos Disciplinarios} \\
		\hline
		\endhead
		% Plan de Febrero:
		Febrero &
		Computadoras. Concepto y elementos constituyentes.\par
		Sistema Operativo. Concepto, ejemplos y funciones.\par
		Periféricos. Concepto, tipos, uso y ejemplos.\par
		Archivos y programas. Concepto, tipos, usos y ejemplos. &
		\begin{itemize}
			\item Utilizar con normalidad una computadora en cualquiera de sus formatos (PC, Móvil, Portátil, etc).
		\end{itemize} &
		\begin{itemize}
			\item Utilizar correctamente los periféricos más comunes (mouse, teclado, monitor, etc).
			\item Utilizar correctamente los archivos y programas más comunes (imágenes, audio, texto plano, etc).
			\item Comprender el concepto y uso de las computadoras y los sistemas operativos.
			\item Ayudar a sus compañeros a comprender los conceptos o habilidades aprendidas en clase.
			\item Utilizar diversos medios para obtener datos o información (preguntar a personas de confianza, deducciones, experimentación, etc).
			\item Mostrar curiosidad por los elementos de su entorno.
		\end{itemize} &
		Ilustraciones, computadoras, juegos, experimentos sencillos, etc. &
		Preguntas orales, Ejercicios de aplicación, Juegos, etc &
		 - \\
		\hline
		% Plan de Marzo:
		Marzo &
		Lógica de la Programación. Aplicaciones a la vida real y ejemplos. &
		\begin{itemize}
			\item Asimilar los principios básicos de la lógica (PI, PRS, PNC, PTE, PC, etc).
			\item Asociar causas a efectos y viceversa.
			\item Respetar las reglas establecidas en diversas situaciones.
		\end{itemize} &
		\begin{itemize}
			\item Jugar respetando las reglas previamente establecidas.
			\item Planear estrategias para ganar juegos.
			\item Ejecutar estrategias propuestas, tanto por sí mismo como por el profesor, para ganar juegos.
			\item Ayudar a sus compañeros a comprender los conceptos o habilidades aprendidas en clase.
			\item Elaborar programas que resuelvan problemas simples presentados en su entorno.
			\item Aplicar las instrucciones de un programa propuesto a la resolución de problemáticas sencillas presentadas en su día a día.
		\end{itemize} &
		Puzzles, Acertijos, Aplicaciones Web, Juegos, Experimentos sencillos, etc &
		Preguntas orales, Ejercicios de aplicación, Juegos, etc. &
		"¡Jugando también aprendo!"\par
		El proyecto consiste en hacer que los chicos que aún no pueden leer o escribir también puedan ejercitar su razonamiento lógico. 
		La mejor manera que se me ocurre para lograrlo es a través de los juegos, pues revelan mucho acerca del alumno, y permiten inculcarle 
		principios de la lógica, como el Principio de Identidad, estructuras de la programación, como secuencias, condicionales y bucles, 
		y el concepto que domina la programción actual: el objeto. \\
		\hline
		% Plan de Abril:
		Abril &
		Lógica de la Programación. Secuencia. &
		\begin{itemize}
			\item Comprender el concepto de secuencialidad.
			\item Aplicar el concepto de secuencialidad a situaciones diarias que así lo demanden.
			\item Asimilar los principios básicos de la lógica (PI, PRS, PNC, PTE, PC, etc).
			\item Asociar causas a efectos y viceversa.
			\item Respetar las reglas establecidas en diversas situaciones.
		\end{itemize} &
		\begin{itemize}
			\item Explicar secuencias sencillas de su día a día.
			\item Ordenar ideas expuestas mediante un razonamiento lógico.
			\item Jugar respetando las reglas previamente establecidas.
			\item Ayudar a sus compañeros a comprender los conceptos y/o habilidades aprendidas en clase.
			\item Elaborar programas que resuelvan problemas simples presentados en su entorno.
			\item Aplicar las instrucciones de un programa propuesto a la resolución de problemáticas sencillas presentadas en su día a día.
		\end{itemize} &
		Puzzles, Acertijos, Aplicaciones Web, Juegos, Experimentos sencillos, Libros de Cuentos, etc &
		Preguntas orales, Ejercicios de aplicación, Juegos, etc. &
		"¡Jugando también aprendo!"\par
		El proyecto consiste en hacer que los chicos que aún no pueden leer o escribir también puedan ejercitar su razonamiento lógico. 
		La mejor manera que se me ocurre para lograrlo es a través de los juegos, pues revelan mucho acerca del alumno, y permiten inculcarle 
		principios de la lógica, como el Principio de Identidad, estructuras de la programación, como secuencias, condicionales y bucles, 
		y el concepto que domina la programción actual: el objeto. \\
		\hline
		% Plan de Mayo:
		Mayo &
		Lógica de la Programación. Condicionales. &
		\begin{itemize}
			\item Comprender el concepto de condicionalidad.
			\item Aplicar el concepto de condicionalidad a situaciones diarias que así lo demanden.
			\item Asimilar los principios básicos de la lógica (PI, PRS, PNC, PTE, PC, etc).
			\item Asociar causas a efectos y viceversa.
			\item Respetar las reglas establecidas en diversas situaciones.
		\end{itemize} &
		\begin{itemize}
			\item Explicar situaciones sencillas de su día a día mediante relaciones condición-acción.
			\item Ordenar ideas expuestas mediante un razonamiento lógico.
			\item Asocia condiciones y conclusiones mediante lógicamente.
			\item Jugar respetando las reglas previamente establecidas.
			\item Ayudar a sus compañeros a comprender los conceptos y/o habilidades aprendidas en clase.
			\item Elaborar programas que resuelvan problemas simples presentados en su entorno.
			\item Aplicar las instrucciones de un programa propuesto a la resolución de problemáticas sencillas presentadas en su día a día.
		\end{itemize} &
		Puzzles, Acertijos, Aplicaciones Web, Juegos, Experimentos sencillos, Libros de Cuentos, etc &
		Preguntas orales, Ejercicios de aplicación, Juegos, etc. &
		"¡Jugando también aprendo!"\par
		El proyecto consiste en hacer que los chicos que aún no pueden leer o escribir también puedan ejercitar su razonamiento lógico. 
		La mejor manera que se me ocurre para lograrlo es a través de los juegos, pues revelan mucho acerca del alumno, y permiten inculcarle 
		principios de la lógica, como el Principio de Identidad, estructuras de la programación, como secuencias, condicionales y bucles, 
		y el concepto que domina la programción actual: el objeto. \\
		\hline
		% Plan de Junio:
		Junio &
		Exámenes. Evaluación. &
		\begin{itemize}
			\item Asimilar los principios básicos de la lógica (PI, PRS, PNC, PTE, PC, etc).
			\item Asociar causas a efectos y viceversa.
			\item Respetar las reglas establecidas en diversas situaciones.
		\end{itemize} &
		\begin{itemize}
			\item Jugar respetando las reglas previamente establecidas.
			\item Planear estrategias para ganar juegos.
			\item Ejecutar estrategias propuestas, tanto por sí mismo como por el profesor, para ganar juegos.
			\item Ayudar a sus compañeros a comprender los conceptos y/o habilidades aprendidas en clase.
			\item Elaborar programas que resuelvan problemas simples presentados en su entorno.
			\item Aplicar las instrucciones de un programa propuesto a la resolución de problemáticas sencillas presentadas en su día a día.
 		\end{itemize} &
		Ejercicios de evaluación, Preguntas acerca de conceptos clave, Sketches. &
		Ejercicios de evaluación, Preguntas acerca de conceptos clave, Sketches. &
		- \\
		\hline
		% Plan de Julio:
		Julio &
		Lógica de la Programación. Aplicación y práctica. &
		\begin{itemize}
			\item Asimilar los principios básicos de la lógica (PI, PRS, PNC, PTE, PC, etc).
			\item Asociar causas a efectos y viceversa.
			\item Respetar las reglas establecidas en diversas situaciones.
		\end{itemize} &
		\begin{itemize}
			\item Jugar respetando las reglas previamente establecidas.
			\item Planear estrategias para ganar juegos.
			\item Ejecutar estrategias propuestas, tanto por sí mismo como por el profesor, para ganar juegos.
			\item Ayudar a sus compañeros a comprender los conceptos y/o habilidades aprendidas en clase.
			\item Elaborar programas que resuelvan problemas simples presentados en su entorno.
			\item Aplicar las instrucciones de un programa propuesto a la resolución de problemáticas sencillas presentadas en su día a día.
 		\end{itemize} &
		Ejercicios de evaluación, Preguntas acerca de conceptos clave, Sketches. &
		Ejercicios de evaluación, Preguntas acerca de conceptos clave, Sketches. &
		- \\
		\hline
		% Plan de Agosto:
		Agosto &
		Lógica de la Programación. Bucles. &
		\begin{itemize}
			\item Comprender el concepto de repetición.
			\item Aplicar el concepto de repetición a situaciones diarias que así lo demanden.
			\item Asimilar los principios básicos de la lógica (PI, PRS, PNC, PTE, PC, etc).
			\item Asociar causas a efectos y viceversa.
			\item Respetar las reglas establecidas en diversas situaciones.
		\end{itemize} &
		\begin{itemize}
			\item Explicar situaciones cotidianas de su día a día aplicando el concepto de repetición.
			\item Clasificar los bucles según la cantidad de veces que se repiten.
			\item Hallar patrones en secuencias previamente expuestas.
			\item Detectar la condición de repetición de acontecimientos concretos.
			\item Jugar respetando las reglas previamente establecidas.
			\item Ayudar a sus compañeros a comprender los conceptos y/o habilidades aprendidas en clase.
			\item Elaborar programas que resuelvan problemas simples presentados en su entorno.
			\item Aplicar las instrucciones de un programa propuesto a la resolución de problemáticas sencillas presentadas en su día a día.
		\end{itemize} &
		Ilustraciones, computadoras, juegos, experimentos sencillos, etc. &
		Preguntas orales, Ejercicios de aplicación, Juegos, etc &
		"¡Jugando también aprendo!"\par
		El proyecto consiste en hacer que los chicos que aún no pueden leer o escribir también puedan ejercitar su razonamiento lógico. 
		La mejor manera que se me ocurre para lograrlo es a través de los juegos, pues revelan mucho acerca del alumno, y permiten inculcarle 
		principios de la lógica, como el Principio de Identidad, estructuras de la programación, como secuencias, condicionales y bucles, 
		y el concepto que domina la programción actual: el objeto. \\
		\hline
		% Plan de Septiembre:
		Septiembre &
		Lógica de la Programación. Bucle de Eventos. &
		\begin{itemize}
			\item Comprender el concepto de evento.
			\item Aplicar el concepto de evento a situaciones que así lo requieran.
			\item Asimilar los principios básicos de la lógica (PI, PRS, PNC, PTE, PC, etc).
			\item Asociar causas a efectos y viceversa.
			\item Respetar las reglas establecidas en diversas situaciones.
		\end{itemize} &
		\begin{itemize}
			\item Manejar múltiples tareas mediante el modelo del bucle de eventos.
			\item Jugar respetando las reglas previamente establecidas.
			\item Ayudar a sus compañeros a comprender los conceptos y/o habilidades aprendidas en clase.
			\item Elaborar programas que resuelvan problemas simples presentados en su entorno.
			\item Aplicar las instrucciones de un programa propuesto a la resolución de problemáticas sencillas presentadas en su día a día.
		\end{itemize} &
		Ilustraciones, computadoras, juegos, experimentos sencillos, etc. &
		Preguntas orales, Ejercicios de aplicación, Juegos, etc &
		"¡Jugando también aprendo!"\par
		El proyecto consiste en hacer que los chicos que aún no pueden leer o escribir también puedan ejercitar su razonamiento lógico. 
		La mejor manera que se me ocurre para lograrlo es a través de los juegos, pues revelan mucho acerca del alumno, y permiten inculcarle 
		principios de la lógica, como el Principio de Identidad, estructuras de la programación, como secuencias, condicionales y bucles, 
		y el concepto que domina la programción actual: el objeto. \\
		\hline
		% Plan de Octubre:
		Octubre &
		Lógica de la Programación. Aplicación y Práctica. &
		\begin{itemize}
			\item Comprender el concepto de repetición.
			\item Aplicar el concepto de repetición a situaciones diarias que así lo demanden.
			\item Asimilar los principios básicos de la lógica (PI, PRS, PNC, PTE, PC, etc).
			\item Asociar causas a efectos y viceversa.
			\item Respetar las reglas establecidas en diversas situaciones.
		\end{itemize} &
		\begin{itemize}
			\item Explicar situaciones cotidianas de su día a día aplicando el concepto de repetición.
			\item Hallar patrones en secuencias previamente expuestas.
			\item Detectar la condición de repetición de acontecimientos concretos.
			\item Jugar respetando las reglas previamente establecidas.
			\item Ayudar a sus compañeros a comprender los conceptos y/o habilidades aprendidas en clase.
			\item Elaborar programas que resuelvan problemas simples presentados en su entorno.
			\item Aplicar las instrucciones de un programa propuesto a la resolución de problemáticas sencillas presentadas en su día a día.
		\end{itemize} &
		Ejercicios de evaluación, Preguntas acerca de conceptos clave, Sketches. &
		Ejercicios de evaluación, Preguntas acerca de conceptos clave, Sketches. &
		"¡Jugando también aprendo!"\par
		El proyecto consiste en hacer que los chicos que aún no pueden leer o escribir también puedan ejercitar su razonamiento lógico. 
		La mejor manera que se me ocurre para lograrlo es a través de los juegos, pues revelan mucho acerca del alumno, y permiten inculcarle 
		principios de la lógica, como el Principio de Identidad, estructuras de la programación, como secuencias, condicionales y bucles, 
		y el concepto que domina la programción actual: el objeto. \\
		\hline
		% Plan de Noviembre:
		Noviembre &
		Lógica de la Programación. Aplicación y práctica. &
		\begin{itemize}
			\item Asimilar los principios básicos de la lógica (PI, PRS, PNC, PTE, PC, etc).
			\item Asociar causas a efectos y viceversa.
			\item Respetar las reglas establecidas en diversas situaciones.
		\end{itemize} &
		\begin{itemize}
			\item Jugar respetando las reglas previamente establecidas.
			\item Planear estrategias para ganar juegos.
			\item Ejecutar estrategias propuestas, tanto por sí mismo como por el profesor, para ganar juegos.
			\item Ayudar a sus compañeros a comprender los conceptos y/o habilidades aprendidas en clase.
			\item Elaborar programas que resuelvan problemas simples presentados en su entorno.
			\item Aplicar las instrucciones de un programa propuesto a la resolución de problemáticas sencillas presentadas en su día a día.
 		\end{itemize} &
		Ejercicios de evaluación, Preguntas acerca de conceptos clave, Sketches. &
		Ejercicios de evaluación, Preguntas acerca de conceptos clave, Sketches. &
		- \\
		\hline

	\end{longtable}
	\pagebreak[4]
	% ----------------------------------------------------------------------------------------------------------------------------------------
\end{document}
