\documentclass[landscape, a4paper, 10pt]{article}
\usepackage[spanish]{babel}
\usepackage[utf8]{inputenc}
\usepackage{array}
\usepackage{graphicx}
\usepackage{tabularx}
\usepackage{longtable}
\usepackage[a4paper,margin=1.5cm,landscape]{geometry}
\usepackage{marginnote}
\usepackage{rotating}
\usepackage{blindtext}
\usepackage[dvipsnames]{xcolor}
\usepackage{tcolorbox}
\newtcolorbox{bgshadow}{
	arc=0pt,
	boxrule=2pt,
	colback=Apricot,
	width=\textheight,
	halign=center
}
% Tamaño de las celdas:
\newcommand{\smallcellwidth}{0.7in}
\newcommand{\normalcellwidth}{1.2in}
\newcommand{\bigcellwidth}{2.0in}
% Identificación:
\newcommand{\profesor}{Santiago Wu}
\newcommand{\discipline}{Robótica y Programación}
\newcommand{\currentyear}{2023}
\newcommand{\institution}{Centro Educativo y Técnico San Agustín}
\newcommand{\CEGSA}{cegsa-logo.png}
\newcommand{\CETSA}{cetsa-logo.png}
% -
\begin{document}
	% Etiqueta:
	%\reversemarginpar\marginnote{
	%	\begin{turn}{90}
	%		\begin{bgshadow}
	%			Centro Educativo y Técnico San Agustín 2023
	%		\end{bgshadow}
	%	\end{turn}
	%}
	% Cabecera:
	\begin{tabularx}{\textwidth}{ >{\raggedright\arraybackslash}X >{\centering\arraybackslash}X >{\raggedleft\arraybackslash}X }
		\includegraphics[width=0.3\textwidth]{\CEGSA} &
		\textbf{PLAN ANUAL DE CLASES} &
		\includegraphics[width=0.3\textwidth]{\CETSA}
	\end{tabularx}
	% Identificación:
	\begin{tabularx}{\textwidth}{ >{\raggedright\arraybackslash}X >{\raggedright\arraybackslash}X >{\raggedright\arraybackslash}X }
		Docente: \profesor &
		Turno: - &
		Año: \currentyear \\
		Disciplina: \discipline &
		Grado/Curso: 1er Grado &
		 \\
	\end{tabularx}
	% Contenido:
	\centering
	\begin{longtable}{|m{\smallcellwidth}|p{\normalcellwidth}|p{\bigcellwidth}|p{\bigcellwidth}|p{\normalcellwidth}|p{\normalcellwidth}|p{\normalcellwidth}|}
	%\begin{tabularx}{\textwidth}{|r|r|r|r|r|r|r|}
		\hline
		\textbf{Mes} &
		\textbf{Contenido/Unidad Temática} &
		\textbf{Capacidades} &
		\textbf{Indicadores} &
		\textbf{Recursos Didácticos/Uso de TIC's} &
		\textbf{Instrumentos de Evaluación} &
		\textbf{Proyectos Disciplinarios} \\
		\hline
		\endhead
		% Plan de Febrero:
		Febrero &
		Introducción a la Lógica &
		\begin{itemize}
			\item Hallar soluciones creativas a situaciones problemáticas mediante el uso del razonamiento lógico.
			\item Mostrar interés o curiosidad por las cosas de su entorno.
		\end{itemize} &
		\begin{itemize}
			\item Elabora estrategias para ganar diversos tipos de juegos de mesa (Casita robada, UNO, Ajedrez, Laberintos, etc).
			\item Resuelve diversos tipos de acertijos y puzzles correctamente.
			\item Emplea diversos medios para la obtención de información (Preguntas al profesor, los compañeros o los padres, uso de lógica deductiva, experimentación, etc).
 		\end{itemize} &
		Juegos diversos, Acertijos, Puzzles, Libros de cuentos, etc. &
		Cuestionarios Orales, Juegos o resolución de Acertijos y Trivias. &
		 - \\
		\hline
		% Plan de Marzo:
		Marzo &
		Introducción a la Lógica &
		\begin{itemize}
			\item Hallar soluciones creativas a situaciones problemáticas mediante el uso del razonamiento lógico.
			\item Mostrar interés o curiosidad por las cosas de su entorno.
		\end{itemize} &
		\begin{itemize}
			\item Elabora estrategias para ganar diversos tipos de juegos de mesa (Casita robada, UNO, Ajedrez, Laberintos, etc).
			\item Resuelve diversos tipos de acertijos y puzzles correctamente.
			\item Demuestra interés por los experimentos realizados en clase.
			\item Emplea diversos medios para la obtención de información (Preguntas al profesor, los compañeros o los padres, uso de lógica deductiva, experimentación, etc).
 		\end{itemize} &
		Juegos diversos, Acertijos, Puzzles, Libros de cuentos, Experimentos sencillos, etc. &
		Cuestionarios Orales y/o escritos, Juegos o resolución de Acertijos y Trivias. &
		"Llegó la caballería"\par
		Este proyecto parte del hecho de que los chicos de 1er Grado aún no poseen habilides básicas, como leer y escribir, aritmética básica, entre
		otras. Por este motivo mi propuesta en el marco del área de \discipline  es ayudarlos a adquirir habilidades básicas, como el 
		pensamiento crítico, el análisis y resolución de situaciones problemáticas, y el razonamiento lógico, ayudándolos en sus 
		necesidades con otras materias mediante el empleo de juegos de mesa, acertijos y experimentos sencillos, de modo que puedan 
		sentirse motivados constantemente a investigar y aprender más sobre el mundo que los rodea.\par
		Así pues, considero que mi objetivo dentro del área de \discipline  está cumplido. \\
		\hline
		% Plan de Abril:
		Abril &
		Introducción a la Programación con "Scratch jr". &
		\begin{itemize}
			\item Hallar soluciones creativas a situaciones problemáticas mediante el uso del razonamiento lógico.
			\item Mostrar interés o curiosidad por las cosas de su entorno.
			\item Utilizar Scratch jr para contar cuentos sencillos.
			\item Utilizar las distintas herramientas tecnológicas a su alcance (PC's, Móviles, etc).
			\item Ordenar ideas de manera lógica y coherente.
			\item Emplear la creación de historias como medio para aprender y/o fijar conocimientos.
			\item Asimilar conceptos básicos de la Programación Estructurada.
			\item Asimilar conceptos básicos de la Programación Orientada a Objetos.
			\item Incluir en sus razonamientos los principios básicos de la lógica (PI, PNC, PRS, PTE, etc).
		\end{itemize} &
		\begin{itemize}
			\item Reconoce los elementos de la interfaz de "Scratch jr".
			\item Realiza acciones básicas de manejo del proyecto (Creación, Guardado, Apertura, Cierre, Eliminación, etc).
			\item Identifica las causas de un efecto y viceversa.
			\item Redacta cuentos acordes a su capacidad de manera coherente y lógica.
			\item Reconoce los elementos básicos de una computadora y/o móvil.
			\item Realiza acciones básicas propias del entorno de programación (Hacer que los distintos objetos se muevan, pasar a la 
			siguiente página, etc).
			\item Utiliza una computadora y/o móvil con normalidad.
			\item Demuestra interés por las actividades realizadas en clase.
			\item Emplea diversos medios para la obtención de información (Preguntas al profesor, los compañeros o los padres, 
			uso de lógica deductiva, experimentación, etc).
 		\end{itemize} &
		Juegos diversos, Acertijos, Puzzles, Libros de cuentos, Experimentos sencillos, "Scratch jr", etc. &
		Cuestionarios Orales, Juegos o resolución de Acertijos y Trivias, Desafíos que incluyan el uso de "Scratch jr.". &
		"Llegó la caballería"\par
		Este proyecto parte del hecho de que los chicos de 1er Grado aún no poseen habilides básicas, como leer y escribir, aritmética básica, entre
		otras. Por este motivo mi propuesta en el marco del área de \discipline  es ayudarlos a adquirir habilidades básicas, como el 
		pensamiento crítico, el análisis y resolución de situaciones problemáticas, y el razonamiento lógico, ayudándolos en sus 
		necesidades con otras materias mediante el empleo de juegos de mesa, acertijos y experimentos sencillos, de modo que puedan 
		sentirse motivados constantemente a investigar y aprender más sobre el mundo que los rodea.\par
		Así pues, considero que mi objetivo dentro del área de \discipline  está cumplido. \\
		\hline
		% Plan de Mayo:
		Mayo &
		Introducción a la Programación con "Scratch jr.". &
		\begin{itemize}
			\item Hallar soluciones creativas a situaciones problemáticas mediante el uso del razonamiento lógico.
			\item Mostrar interés o curiosidad por las cosas de su entorno.
			\item Utilizar Scratch jr para contar cuentos sencillos.
			\item Utilizar las distintas herramientas tecnológicas a su alcance (PC's, Móviles, etc).
			\item Ordenar ideas de manera lógica y coherente.
			\item Emplear la creación de historias como medio para aprender y/o fijar conocimientos.
			\item Asimilar conceptos básicos de la Programación Estructurada.
			\item Asimilar conceptos básicos de la Programación Orientada a Objetos.
			\item Incluir en sus razonamientos los principios básicos de la lógica (PI, PNC, PRS, PTE, etc).
		\end{itemize} &
		\begin{itemize}
			\item Reconoce los elementos de la interfaz de "Scratch jr".
			\item Realiza acciones básicas de manejo del proyecto (Creación, Guardado, Apertura, Cierre, Eliminación, etc).
			\item Identifica las causas de un efecto y viceversa.
			\item Redacta cuentos acordes a su capacidad de manera coherente y lógica.
			\item Reconoce los elementos básicos de una computadora y/o móvil.
			\item Realiza acciones básicas propias del entorno de programación (Hacer que los distintos objetos se muevan, pasar a la 
			siguiente página, etc).
			\item Utiliza una computadora y/o móvil con normalidad.
			\item Demuestra interés por las actividades realizadas en clase.
			\item Emplea diversos medios para la obtención de información (Preguntas al profesor, los compañeros o los padres, 
			uso de lógica deductiva, experimentación, etc).
 		\end{itemize} &
		Juegos diversos, Acertijos, Puzzles, Libros de cuentos, Experimentos sencillos, "Scratch jr", etc. &
		Cuestionarios Orales, Juegos o resolución de Acertijos y Trivias, Desafíos que incluyan el uso de "Scratch jr.". &
		"Llegó la caballería"\par
		Este proyecto parte del hecho de que los chicos de 1er Grado aún no poseen habilides básicas, como leer y escribir, aritmética básica, entre
		otras. Por este motivo mi propuesta en el marco del área de \discipline  es ayudarlos a adquirir habilidades básicas, como el 
		pensamiento crítico, el análisis y resolución de situaciones problemáticas, y el razonamiento lógico, ayudándolos en sus 
		necesidades con otras materias mediante el empleo de juegos de mesa, acertijos y experimentos sencillos, de modo que puedan 
		sentirse motivados constantemente a investigar y aprender más sobre el mundo que los rodea.\par
		Así pues, considero que mi objetivo dentro del área de \discipline  está cumplido. \\
		\hline
		% Plan de Junio:
		Junio &
		Introducción a la Programación con "Scratch jr.". &
		\begin{itemize}
			\item Hallar soluciones creativas a situaciones problemáticas mediante el uso del razonamiento lógico.
			\item Mostrar interés o curiosidad por las cosas de su entorno.
			\item Utilizar Scratch jr para contar cuentos sencillos.
			\item Utilizar las distintas herramientas tecnológicas a su alcance (PC's, Móviles, etc).
			\item Ordenar ideas de manera lógica y coherente.
			\item Emplear la creación de historias como medio para aprender y/o fijar conocimientos.
			\item Asimilar conceptos básicos de la Programación Estructurada.
			\item Asimilar conceptos básicos de la Programación Orientada a Objetos.
			\item Incluir en sus razonamientos los principios básicos de la lógica (PI, PNC, PRS, PTE, etc).
		\end{itemize} &
		\begin{itemize}
			\item Reconoce los elementos de la interfaz de "Scratch jr".
			\item Realiza acciones básicas de manejo del proyecto (Creación, Guardado, Apertura, Cierre, Eliminación, etc).
			\item Identifica las causas de un efecto y viceversa.
			\item Redacta cuentos acordes a su capacidad de manera coherente y lógica.
			\item Reconoce los elementos básicos de una computadora y/o móvil.
			\item Realiza acciones básicas propias del entorno de programación (Hacer que los distintos objetos se muevan, pasar a la 
			siguiente página, etc).
			\item Utiliza una computadora y/o móvil con normalidad.
			\item Demuestra interés por las actividades realizadas en clase.
			\item Emplea diversos medios para la obtención de información (Preguntas al profesor, los compañeros o los padres, 
			uso de lógica deductiva, experimentación, etc).
 		\end{itemize} &
		Juegos diversos, Acertijos, Puzzles, Libros de cuentos, Experimentos sencillos, "Scratch jr", etc. &
		Cuestionarios Orales, Juegos o resolución de Acertijos y Trivias, Desafíos que incluyan el uso de "Scratch jr.". &
		"Llegó la caballería"\par
		Este proyecto parte del hecho de que los chicos de 1er Grado aún no poseen habilides básicas, como leer y escribir, aritmética básica, entre
		otras. Por este motivo mi propuesta en el marco del área de \discipline  es ayudarlos a adquirir habilidades básicas, como el 
		pensamiento crítico, el análisis y resolución de situaciones problemáticas, y el razonamiento lógico, ayudándolos en sus 
		necesidades con otras materias mediante el empleo de juegos de mesa, acertijos y experimentos sencillos, de modo que puedan 
		sentirse motivados constantemente a investigar y aprender más sobre el mundo que los rodea.\par
		Así pues, considero que mi objetivo dentro del área de \discipline  está cumplido. \\
		\hline
		% Plan de Julio:
		Julio &
		Introducción a la Programación con "Scratch jr.". &
		\begin{itemize}
			\item Hallar soluciones creativas a situaciones problemáticas mediante el uso del razonamiento lógico.
			\item Mostrar interés o curiosidad por las cosas de su entorno.
			\item Utilizar Scratch jr para contar cuentos sencillos.
			\item Utilizar las distintas herramientas tecnológicas a su alcance (PC's, Móviles, etc).
			\item Ordenar ideas de manera lógica y coherente.
			\item Emplear la creación de historias como medio para aprender y/o fijar conocimientos.
			\item Asimilar conceptos básicos de la Programación Estructurada.
			\item Asimilar conceptos básicos de la Programación Orientada a Objetos.
			\item Incluir en sus razonamientos los principios básicos de la lógica (PI, PNC, PRS, PTE, etc).
		\end{itemize} &
		\begin{itemize}
			\item Reconoce los elementos de la interfaz de "Scratch jr".
			\item Realiza acciones básicas de manejo del proyecto (Creación, Guardado, Apertura, Cierre, Eliminación, etc).
			\item Identifica las causas de un efecto y viceversa.
			\item Redacta cuentos acordes a su capacidad de manera coherente y lógica.
			\item Reconoce los elementos básicos de una computadora y/o móvil.
			\item Realiza acciones básicas propias del entorno de programación (Hacer que los distintos objetos se muevan, pasar a la 
			siguiente página, etc).
			\item Utiliza una computadora y/o móvil con normalidad.
			\item Demuestra interés por las actividades realizadas en clase.
			\item Emplea diversos medios para la obtención de información (Preguntas al profesor, los compañeros o los padres, 
			uso de lógica deductiva, experimentación, etc).
 		\end{itemize} &
		Juegos diversos, Acertijos, Puzzles, Libros de cuentos, Experimentos sencillos, "Scratch jr", etc. &
		Cuestionarios Orales, Juegos o resolución de Acertijos y Trivias, Desafíos que incluyan el uso de "Scratch jr.". &
		"Llegó la caballería"\par
		Este proyecto parte del hecho de que los chicos de 1er Grado aún no poseen habilides básicas, como leer y escribir, aritmética básica, entre
		otras. Por este motivo mi propuesta en el marco del área de \discipline  es ayudarlos a adquirir habilidades básicas, como el 
		pensamiento crítico, el análisis y resolución de situaciones problemáticas, y el razonamiento lógico, ayudándolos en sus 
		necesidades con otras materias mediante el empleo de juegos de mesa, acertijos y experimentos sencillos, de modo que puedan 
		sentirse motivados constantemente a investigar y aprender más sobre el mundo que los rodea.\par
		Así pues, considero que mi objetivo dentro del área de \discipline  está cumplido. \\
		\hline
		% Plan de Agosto:
		Agosto &
		Introducción a la Programación con "Scratch jr.". &
		\begin{itemize}
			\item Hallar soluciones creativas a situaciones problemáticas mediante el uso del razonamiento lógico.
			\item Mostrar interés o curiosidad por las cosas de su entorno.
			\item Utilizar Scratch jr para contar cuentos sencillos.
			\item Utilizar las distintas herramientas tecnológicas a su alcance (PC's, Móviles, etc).
			\item Ordenar ideas de manera lógica y coherente.
			\item Emplear la creación de historias como medio para aprender y/o fijar conocimientos.
			\item Asimilar conceptos básicos de la Programación Estructurada.
			\item Asimilar conceptos básicos de la Programación Orientada a Objetos.
			\item Incluir en sus razonamientos los principios básicos de la lógica (PI, PNC, PRS, PTE, etc).
		\end{itemize} &
		\begin{itemize}
			\item Reconoce los elementos de la interfaz de "Scratch jr".
			\item Realiza acciones básicas de manejo del proyecto (Creación, Guardado, Apertura, Cierre, Eliminación, etc).
			\item Identifica las causas de un efecto y viceversa.
			\item Redacta cuentos acordes a su capacidad de manera coherente y lógica.
			\item Reconoce los elementos básicos de una computadora y/o móvil.
			\item Realiza acciones básicas propias del entorno de programación (Hacer que los distintos objetos se muevan, pasar a la 
			siguiente página, etc).
			\item Utiliza una computadora y/o móvil con normalidad.
			\item Demuestra interés por las actividades realizadas en clase.
			\item Emplea diversos medios para la obtención de información (Preguntas al profesor, los compañeros o los padres, 
			uso de lógica deductiva, experimentación, etc).
 		\end{itemize} &
		Juegos diversos, Acertijos, Puzzles, Libros de cuentos, Experimentos sencillos, "Scratch jr", etc. &
		Cuestionarios Orales, Juegos o resolución de Acertijos y Trivias, Desafíos que incluyan el uso de "Scratch jr.". &
		"Llegó la caballería"\par
		Este proyecto parte del hecho de que los chicos de 1er Grado aún no poseen habilides básicas, como leer y escribir, aritmética básica, entre
		otras. Por este motivo mi propuesta en el marco del área de \discipline  es ayudarlos a adquirir habilidades básicas, como el 
		pensamiento crítico, el análisis y resolución de situaciones problemáticas, y el razonamiento lógico, ayudándolos en sus 
		necesidades con otras materias mediante el empleo de juegos de mesa, acertijos y experimentos sencillos, de modo que puedan 
		sentirse motivados constantemente a investigar y aprender más sobre el mundo que los rodea.\par
		Así pues, considero que mi objetivo dentro del área de \discipline  está cumplido. \\
		\hline
		% Plan de Septiembre:
		Septiembre &
		Introducción a la Programación con "Scratch jr.". &
		\begin{itemize}
			\item Hallar soluciones creativas a situaciones problemáticas mediante el uso del razonamiento lógico.
			\item Mostrar interés o curiosidad por las cosas de su entorno.
			\item Utilizar Scratch jr para contar cuentos sencillos.
			\item Utilizar las distintas herramientas tecnológicas a su alcance (PC's, Móviles, etc).
			\item Ordenar ideas de manera lógica y coherente.
			\item Emplear la creación de historias como medio para aprender y/o fijar conocimientos.
			\item Asimilar conceptos básicos de la Programación Estructurada.
			\item Asimilar conceptos básicos de la Programación Orientada a Objetos.
			\item Incluir en sus razonamientos los principios básicos de la lógica (PI, PNC, PRS, PTE, etc).
		\end{itemize} &
		\begin{itemize}
			\item Reconoce los elementos de la interfaz de "Scratch jr".
			\item Realiza acciones básicas de manejo del proyecto (Creación, Guardado, Apertura, Cierre, Eliminación, etc).
			\item Identifica las causas de un efecto y viceversa.
			\item Redacta cuentos acordes a su capacidad de manera coherente y lógica.
			\item Reconoce los elementos básicos de una computadora y/o móvil.
			\item Realiza acciones básicas propias del entorno de programación (Hacer que los distintos objetos se muevan, pasar a la 
			siguiente página, etc).
			\item Utiliza una computadora y/o móvil con normalidad.
			\item Demuestra interés por las actividades realizadas en clase.
			\item Emplea diversos medios para la obtención de información (Preguntas al profesor, los compañeros o los padres, 
			uso de lógica deductiva, experimentación, etc).
 		\end{itemize} &
		Juegos diversos, Acertijos, Puzzles, Libros de cuentos, Experimentos sencillos, "Scratch jr", etc. &
		Cuestionarios Orales, Juegos o resolución de Acertijos y Trivias, Desafíos que incluyan el uso de "Scratch jr.". &
		"Llegó la caballería"\par
		Este proyecto parte del hecho de que los chicos de 1er Grado aún no poseen habilides básicas, como leer y escribir, aritmética básica, entre
		otras. Por este motivo mi propuesta en el marco del área de \discipline  es ayudarlos a adquirir habilidades básicas, como el 
		pensamiento crítico, el análisis y resolución de situaciones problemáticas, y el razonamiento lógico, ayudándolos en sus 
		necesidades con otras materias mediante el empleo de juegos de mesa, acertijos y experimentos sencillos, de modo que puedan 
		sentirse motivados constantemente a investigar y aprender más sobre el mundo que los rodea.\par
		Así pues, considero que mi objetivo dentro del área de \discipline  está cumplido. \\
		\hline
		% Plan de Octubre:
		Octubre &
		Introducción a la Programación con "Scratch jr.". &
		\begin{itemize}
			\item Hallar soluciones creativas a situaciones problemáticas mediante el uso del razonamiento lógico.
			\item Mostrar interés o curiosidad por las cosas de su entorno.
			\item Utilizar Scratch jr para contar cuentos sencillos.
			\item Utilizar las distintas herramientas tecnológicas a su alcance (PC's, Móviles, etc).
			\item Ordenar ideas de manera lógica y coherente.
			\item Emplear la creación de historias como medio para aprender y/o fijar conocimientos.
			\item Asimilar conceptos básicos de la Programación Estructurada.
			\item Asimilar conceptos básicos de la Programación Orientada a Objetos.
			\item Incluir en sus razonamientos los principios básicos de la lógica (PI, PNC, PRS, PTE, etc).
		\end{itemize} &
		\begin{itemize}
			\item Reconoce los elementos de la interfaz de "Scratch jr".
			\item Realiza acciones básicas de manejo del proyecto (Creación, Guardado, Apertura, Cierre, Eliminación, etc).
			\item Identifica las causas de un efecto y viceversa.
			\item Redacta cuentos acordes a su capacidad de manera coherente y lógica.
			\item Reconoce los elementos básicos de una computadora y/o móvil.
			\item Realiza acciones básicas propias del entorno de programación (Hacer que los distintos objetos se muevan, pasar a la 
			siguiente página, etc).
			\item Utiliza una computadora y/o móvil con normalidad.
			\item Demuestra interés por las actividades realizadas en clase.
			\item Emplea diversos medios para la obtención de información (Preguntas al profesor, los compañeros o los padres, 
			uso de lógica deductiva, experimentación, etc).
 		\end{itemize} &
		Juegos diversos, Acertijos, Puzzles, Libros de cuentos, Experimentos sencillos, "Scratch jr", etc. &
		Cuestionarios Orales, Juegos o resolución de Acertijos y Trivias, Desafíos que incluyan el uso de "Scratch jr.". &
		"Llegó la caballería"\par
		Este proyecto parte del hecho de que los chicos de 1er Grado aún no poseen habilides básicas, como leer y escribir, aritmética básica, entre
		otras. Por este motivo mi propuesta en el marco del área de \discipline  es ayudarlos a adquirir habilidades básicas, como el 
		pensamiento crítico, el análisis y resolución de situaciones problemáticas, y el razonamiento lógico, ayudándolos en sus 
		necesidades con otras materias mediante el empleo de juegos de mesa, acertijos y experimentos sencillos, de modo que puedan 
		sentirse motivados constantemente a investigar y aprender más sobre el mundo que los rodea.\par
		Así pues, considero que mi objetivo dentro del área de \discipline  está cumplido. \\
		\hline
		% Plan de Noviembre:
		Noviembre &
		Introducción a la Programación con "Scratch jr.". &
		\begin{itemize}
			\item Hallar soluciones creativas a situaciones problemáticas mediante el uso del razonamiento lógico.
			\item Mostrar interés o curiosidad por las cosas de su entorno.
			\item Utilizar Scratch jr para contar cuentos sencillos.
			\item Utilizar las distintas herramientas tecnológicas a su alcance (PC's, Móviles, etc).
			\item Ordenar ideas de manera lógica y coherente.
			\item Emplear la creación de historias como medio para aprender y/o fijar conocimientos.
			\item Asimilar conceptos básicos de la Programación Estructurada.
			\item Asimilar conceptos básicos de la Programación Orientada a Objetos.
			\item Incluir en sus razonamientos los principios básicos de la lógica (PI, PNC, PRS, PTE, etc).
		\end{itemize} &
		\begin{itemize}
			\item Reconoce los elementos de la interfaz de "Scratch jr".
			\item Realiza acciones básicas de manejo del proyecto (Creación, Guardado, Apertura, Cierre, Eliminación, etc).
			\item Identifica las causas de un efecto y viceversa.
			\item Redacta cuentos acordes a su capacidad de manera coherente y lógica.
			\item Reconoce los elementos básicos de una computadora y/o móvil.
			\item Realiza acciones básicas propias del entorno de programación (Hacer que los distintos objetos se muevan, pasar a la 
			siguiente página, etc).
			\item Utiliza una computadora y/o móvil con normalidad.
			\item Demuestra interés por las actividades realizadas en clase.
			\item Emplea diversos medios para la obtención de información (Preguntas al profesor, los compañeros o los padres, 
			uso de lógica deductiva, experimentación, etc).
 		\end{itemize} &
		Juegos diversos, Acertijos, Puzzles, Libros de cuentos, Experimentos sencillos, "Scratch jr", etc. &
		Cuestionarios Orales, Juegos o resolución de Acertijos y Trivias, Desafíos que incluyan el uso de "Scratch jr.". &
		"Llegó la caballería"\par
		Este proyecto parte del hecho de que los chicos de 1er Grado aún no poseen habilides básicas, como leer y escribir, aritmética básica, entre
		otras. Por este motivo mi propuesta en el marco del área de \discipline  es ayudarlos a adquirir habilidades básicas, como el 
		pensamiento crítico, el análisis y resolución de situaciones problemáticas, y el razonamiento lógico, ayudándolos en sus 
		necesidades con otras materias mediante el empleo de juegos de mesa, acertijos y experimentos sencillos, de modo que puedan 
		sentirse motivados constantemente a investigar y aprender más sobre el mundo que los rodea.\par
		Así pues, considero que mi objetivo dentro del área de \discipline  está cumplido. \\
		\hline

	\end{longtable}
	\pagebreak[4]
	% ----------------------------------------------------------------------------------------------------------------------------------------
	% Outline:
	% - El fin último de la educación, es brindar a los chicos herramientas para enfrentar el mundo en el que viven.
	% - La educación debe cambiar. El uso de la tiza y el pizarrón van casi desapareciendo.
	% - 
	% Cabecera:
	\begin{tabularx}{\textwidth}{ >{\raggedright\arraybackslash}X >{\centering\arraybackslash}X >{\raggedleft\arraybackslash}X }
		\includegraphics[width=0.3\textwidth]{\CEGSA} &
		\textbf{PLAN ANUAL DE CLASES} &
		\includegraphics[width=0.3\textwidth]{\CETSA}
	\end{tabularx}
	% Identificación:
	\begin{tabularx}{\textwidth}{ >{\raggedright\arraybackslash}X >{\raggedright\arraybackslash}X >{\raggedright\arraybackslash}X }
		Docente: \profesor &
		Turno: - &
		Año: \currentyear \\
		Disciplina: \discipline &
		Grado/Curso: 2do Grado &
		 \\
	\end{tabularx}
	% Contenido:
	\centering
	\begin{longtable}{|m{\smallcellwidth}|p{\normalcellwidth}|p{\bigcellwidth}|p{\bigcellwidth}|p{\normalcellwidth}|p{\normalcellwidth}|p{\normalcellwidth}|}
	%\begin{tabularx}{\textwidth}{|r|r|r|r|r|r|r|}
		\hline
		\textbf{Mes} &
		\textbf{Contenido/Unidad Temática} &
		\textbf{Capacidades} &
		\textbf{Indicadores} &
		\textbf{Recursos Didácticos/Uso de TIC's} &
		\textbf{Instrumentos de Evaluación} &
		\textbf{Proyectos Disciplinarios} \\
		\hline
		\endhead
		% Plan de Febrero:
		Febrero &
		Introducción a la Lógica &
		\begin{itemize}
			\item Hallar soluciones creativas a situaciones problemáticas mediante el uso del razonamiento lógico.
			\item Mostrar interés o curiosidad por las cosas de su entorno.
		\end{itemize} &
		\begin{itemize}
			\item Elabora estrategias para ganar diversos tipos de juegos de mesa (Casita robada, UNO, Ajedrez, Laberintos, etc).
			\item Resuelve diversos tipos de acertijos y puzzles correctamente.
			\item Emplea diversos medios para la obtención de información (Preguntas al profesor, los compañeros o los padres, uso 
			del internet, uso de lógica deductiva, experimentación, etc).
 		\end{itemize} &
		Juegos diversos, Acertijos, Puzzles, Libros de cuentos, etc. &
		Cuestionarios Orales, Juegos o resolución de Acertijos y Trivias. &
		 - \\
		\hline
		% Plan de Marzo:
		Marzo &
		Introducción a la Programación y la Algoritmia &
		\begin{itemize}
			\item Analizar problemas simples de su entorno aplicando el razonamiento lógico.
			\item Resuelver problemas sencillos de su medio a través del razonamiento lógico.
			\item Mostrar curiosidad por las cosas de su entorno.
		\end{itemize} &
		\begin{itemize}
			\item Comprende y analiza problemas sencillos de su entorno mediante el razonamiento lógico.
			\item Elabora y propone soluciones creativas a problemas sencillos de su entorno.
			\item Resuelve problemas mediante la ejecución de las soluciones propuestas, tanto por él mismo, como por el profesor.
			\item Comprende principios básicos de lógica (PNC, PI, PTE, PRS, etc).
			\item Identifica instrucciones sencillas propias de la programación (Leer, Imprimir, Crear variable, Guardar en variable, etc).
			\item Emplea diversos medios para la obtención de información (Preguntas al profesor, los compañeros o los padres, 
			uso de lógica deductiva, experimentación, etc).
		\end{itemize} &
		Diagramas ilustrativos, experimentos sencillos, presentaciones, "Scratch jr.", etc.  &
		Ejercicios prácticos, Cuestionarios Orales y/o Escritos, Juegos, etc. &
		"¡Jugando también aprendo!"\par
		Este proyecto parte de la idea de que 'todo problema puede ser puesto bajo un modelo de datos e información'. 
		El mismo consiste en, a través de diversos juegos, hacer que los chicos puedan crear los suyos propios, estableciendo 
		reglas claras y sencillas, y tomando como temas alguno de su propio interés, de modo que en el proceso fije u obtenga 
		nuevos conocimientos acerca del mismo. \\
		\hline
		% Plan de Abril:
		Abril &
		Introducción al entorno de desarrollo "Scratch jr." &
		\begin{itemize}
			\item Analizar problemas simples de su entorno aplicando el razonamiento lógico.
			\item Resuelver problemas sencillos de su medio mediante el uso del razonamiento lógico.
			\item Mostrar curiosidad por las cosas de su entorno.
			\item Utiliza correctamente "Scratch jr." para ejecutar soluciones previamente planteadas, tanto por el profesor, como él mismo.
			\item Usar una computadora y/o móvil con normalidad.
		\end{itemize} &
		\begin{itemize}
			\item Comprende y analiza problemas sencillos de su entorno mediante el razonamiento lógico.
			\item Elabora y propone soluciones creativas a problemas sencillos de su entorno.
			\item Resuelve problemas mediante la ejecución de las soluciones propuestas, tanto por él mismo, como por el profesor.
			\item Reconoce diversos tipos de intrucciones y sus usos en el entorno de "Scratch jr.".
			\item Relaciona el "paso a paso" de una solución propuesta con instrucciones de "Scratch jr.".
			\item Asimila principios básicos de lógica (PNC, PI, PTE, PRS, PC, etc).
			\item Asimila estructuras lógicas simples (Conjunción, Disyunción inclusiva, Bicondicional, Condicional, etc).
			\item Emplea diversos medios para la obtención de información (Preguntas al profesor, los compañeros o los padres, 
			uso de lógica deductiva, experimentación, etc).
		\end{itemize} &
		Diagramas ilustrativos, experimentos sencillos, presentaciones, "Scratch jr.", etc.  &
		Ejercicios prácticos, Cuestionarios Orales y/o Escritos, Juegos, etc. &
		"¡Jugando también aprendo!"\par
		Este proyecto parte de la idea de que 'todo problema puede ser puesto bajo un modelo de datos e información'. 
		El mismo consiste en, a través de diversos juegos, hacer que los chicos puedan crear los suyos propios, estableciendo 
		reglas claras y sencillas, y tomando como temas alguno de su propio interés, de modo que en el proceso fije u obtenga 
		nuevos conocimientos acerca del mismo. \\
		\hline
		% Plan de Mayo:
		Mayo &
		Programación con "Scratch jr." &
		\begin{itemize}
			\item Analizar problemas simples de su entorno aplicando el razonamiento lógico.
			\item Resuelver problemas sencillos de su medio mediante el uso del razonamiento lógico.
			\item Mostrar curiosidad por las cosas de su entorno.
			\item Usar una computadora y/o móvil con normalidad.
			\item Utilizar correctamente "Scratch jr." para ejecutar soluciones previamente planteadas, tanto por el profesor, como él mismo.
			\item Crear historias y/o videojuegos mediante el uso de "Scratch jr." basados en temas de su interés.
			\item Fijar o adquirir conocimientos mediante la creación de historias y juegos con "Scratch jr.".
		\end{itemize} &
		\begin{itemize}
			\item Comprende y analiza problemas sencillos de su entorno mediante el razonamiento lógico.
			\item Elabora y propone soluciones creativas a problemas sencillos de su entorno.
			\item Resuelve problemas mediante la ejecución de las soluciones propuestas, tanto por él mismo, como por el profesor.
			\item Reconoce diversos tipos de intrucciones y sus usos en el entorno de "Scratch jr.".
			\item Relaciona el "paso a paso" de una solución propuesta con instrucciones de "Scratch jr.".
			\item Asimila principios básicos de lógica (PNC, PI, PTE, PRS, PC, etc).
			\item Asimila estructuras lógicas simples (Conjunción, Disyunción inclusiva, Bicondicional, Condicional, etc).
			\item Ordena ideas de manera coherente y cohesiva.
			\item Emplea diversos medios para la obtención de información (Preguntas al profesor, los compañeros o los padres, 
			uso de lógica deductiva, experimentación, etc).
		\end{itemize} &
		Diagramas ilustrativos, experimentos sencillos, presentaciones, "Scratch jr.", etc.  &
		Ejercicios prácticos, Cuestionarios Orales y/o Escritos, Juegos, etc. &
		"¡Jugando también aprendo!"\par
		Este proyecto parte de la idea de que 'todo problema puede ser puesto bajo un modelo de datos e información'. 
		El mismo consiste en, a través de diversos juegos, hacer que los chicos puedan crear los suyos propios, estableciendo 
		reglas claras y sencillas, y tomando como temas alguno de su propio interés, de modo que en el proceso fije u obtenga 
		nuevos conocimientos acerca del mismo. \\
		\hline
		% Plan de Junio:
		Junio &
		Programación con "Scratch jr." &
		\begin{itemize}
			\item Analizar problemas simples de su entorno aplicando el razonamiento lógico.
			\item Resuelver problemas sencillos de su medio mediante el uso del razonamiento lógico.
			\item Mostrar curiosidad por las cosas de su entorno.
			\item Usar una computadora y/o móvil con normalidad.
		\end{itemize} &
		\begin{itemize}
			\item Comprende y analiza problemas sencillos de su entorno mediante el razonamiento lógico.
			\item Elabora y propone soluciones creativas a problemas sencillos de su entorno.
			\item Resuelve problemas mediante la ejecución de las soluciones propuestas, tanto por él mismo, como por el profesor.
			\item Emplea diversos medios para la obtención de información (Preguntas al profesor, los compañeros o los padres, 
			uso de lógica deductiva, experimentación, etc).
		\end{itemize} &
		Diagramas ilustrativos, experimentos sencillos, presentaciones, "Scratch jr.", etc.  &
		Ejercicios prácticos, Cuestionarios Orales y/o Escritos, Juegos, etc. &
		"¡Jugando también aprendo!"\par
		Este proyecto parte de la idea de que 'todo problema puede ser puesto bajo un modelo de datos e información'. 
		El mismo consiste en, a través de diversos juegos, hacer que los chicos puedan crear los suyos propios, estableciendo 
		reglas claras y sencillas, y tomando como temas alguno de su propio interés, de modo que en el proceso fije u obtenga 
		nuevos conocimientos acerca del mismo. \\
		\hline
		% Plan de Julio:
		Julio &
		Programación con "Scratch jr." &
		\begin{itemize}
			\item Analizar problemas simples de su entorno aplicando el razonamiento lógico.
			\item Resuelver problemas sencillos de su medio mediante el uso del razonamiento lógico.
			\item Mostrar curiosidad por las cosas de su entorno.
		\end{itemize} &
		\begin{itemize}
			\item Comprende y analiza problemas sencillos de su entorno mediante el razonamiento lógico.
			\item Elabora y propone soluciones creativas a problemas sencillos de su entorno.
			\item Resuelve problemas mediante la ejecución de las soluciones propuestas, tanto por él mismo, como por el profesor.
			\item Emplea diversos medios para la obtención de información (Preguntas al profesor, los compañeros o los padres, 
			uso de lógica deductiva, experimentación, etc).
		\end{itemize} &
		Diagramas ilustrativos, experimentos sencillos, presentaciones, "Scratch jr.", etc.  &
		Ejercicios prácticos, Cuestionarios Orales y/o Escritos, Juegos, etc. &
		"¡Jugando también aprendo!"\par
		Este proyecto parte de la idea de que 'todo problema puede ser puesto bajo un modelo de datos e información'. 
		El mismo consiste en, a través de diversos juegos, hacer que los chicos puedan crear los suyos propios, estableciendo 
		reglas claras y sencillas, y tomando como temas alguno de su propio interés, de modo que en el proceso fije u obtenga 
		nuevos conocimientos acerca del mismo. \\
		\hline
		% Plan de Agosto:
		Agosto &
		Programación con "Scratch jr." &
		\begin{itemize}
			\item Analizar problemas simples de su entorno aplicando el razonamiento lógico.
			\item Resuelver problemas sencillos de su medio mediante el uso del razonamiento lógico.
			\item Mostrar curiosidad por las cosas de su entorno.
			\item Utilizar correctamente "Scratch jr." para ejecutar soluciones previamente planteadas, tanto por el profesor, como él mismo.
			\item Crear historias y/o videojuegos mediante el uso de "Scratch jr." basados en temas de su interés.
			\item Fijar o adquirir conocimientos mediante la creación de historias y juegos con "Scratch jr.".
			\item Usar una computadora o móvil con normalidad.
		\end{itemize} &
		\begin{itemize}
			\item Comprende y analiza problemas sencillos de su entorno mediante el razonamiento lógico.
			\item Elabora y propone soluciones creativas a problemas sencillos de su entorno.
			\item Resuelve problemas mediante la ejecución de las soluciones propuestas, tanto por él mismo, como por el profesor.
			\item Reconoce diversos tipos de intrucciones y sus usos en el entorno de "Scratch jr.".
			\item Relaciona el "paso a paso" de una solución propuesta con instrucciones de "Scratch jr.".
			\item Asimila principios básicos de lógica (PNC, PI, PTE, PRS, PC, etc).
			\item Asimila estructuras lógicas simples (Conjunción, Disyunción inclusiva, Bicondicional, Condicional, etc).
			\item Entiende las conexiones entre causas y efectos y viceversa.
			\item Ordena ideas de manera coherente y cohesiva.
			\item Emplea diversos medios para la obtención de información (Preguntas al profesor, los compañeros o los padres, 
			uso de lógica deductiva, experimentación, etc).
		\end{itemize} &
		Diagramas ilustrativos, experimentos sencillos, presentaciones, "Scratch jr.", etc.  &
		Ejercicios prácticos, Cuestionarios Orales y/o Escritos, Juegos, etc. &
		"¡Jugando también aprendo!"\par
		Este proyecto parte de la idea de que 'todo problema puede ser puesto bajo un modelo de datos e información'. 
		El mismo consiste en, a través de diversos juegos, hacer que los chicos puedan crear los suyos propios, estableciendo 
		reglas claras y sencillas, y tomando como temas alguno de su propio interés, de modo que en el proceso fije u obtenga 
		nuevos conocimientos acerca del mismo. \\
		\hline
		% Plan de Septiembre:
		Septiembre &
		Programación con "Scratch jr." &
		\begin{itemize}
			\item Analizar problemas simples de su entorno aplicando el razonamiento lógico.
			\item Resuelver problemas sencillos de su medio mediante el uso del razonamiento lógico.
			\item Mostrar curiosidad por las cosas de su entorno.
			\item Utilizar correctamente "Scratch jr." para ejecutar soluciones previamente planteadas, tanto por el profesor, como él mismo.
			\item Crear historias y/o videojuegos mediante el uso de "Scratch jr." basados en temas de su interés.
			\item Fijar o adquirir conocimientos mediante la creación de historias y juegos con "Scratch jr.".
			\item Usar una computadora o móvil con normalidad.
		\end{itemize} &
		\begin{itemize}
			\item Comprende y analiza problemas sencillos de su entorno mediante el razonamiento lógico.
			\item Elabora y propone soluciones creativas a problemas sencillos de su entorno.
			\item Resuelve problemas mediante la ejecución de las soluciones propuestas, tanto por él mismo, como por el profesor.
			\item Reconoce diversos tipos de intrucciones y sus usos en el entorno de "Scratch jr.".
			\item Relaciona el "paso a paso" de una solución propuesta con instrucciones de "Scratch jr.".
			\item Asimila principios básicos de lógica (PNC, PI, PTE, PRS, PC, etc).
			\item Asimila estructuras lógicas simples (Conjunción, Disyunción inclusiva, Bicondicional, Condicional, etc).
			\item Entiende las conexiones entre causas y efectos y viceversa.
			\item Ordena ideas de manera coherente y cohesiva.
			\item Emplea diversos medios para la obtención de información (Preguntas al profesor, los compañeros o los padres, 
			uso de lógica deductiva, experimentación, etc).
		\end{itemize} &
		Diagramas ilustrativos, experimentos sencillos, presentaciones, "Scratch jr.", etc.  &
		Ejercicios prácticos, Cuestionarios Orales y/o Escritos, Juegos, etc. &
		"¡Jugando también aprendo!"\par
		Este proyecto parte de la idea de que 'todo problema puede ser puesto bajo un modelo de datos e información'. 
		El mismo consiste en, a través de diversos juegos, hacer que los chicos puedan crear los suyos propios, estableciendo 
		reglas claras y sencillas, y tomando como temas alguno de su propio interés, de modo que en el proceso fije u obtenga 
		nuevos conocimientos acerca del mismo. \\
		\hline
		% Plan de Octubre:
		Octubre &
		Programación con "Scratch jr." &
		\begin{itemize}
			\item Analizar problemas simples de su entorno aplicando el razonamiento lógico.
			\item Resuelver problemas sencillos de su medio mediante el uso del razonamiento lógico.
			\item Mostrar curiosidad por las cosas de su entorno.
			\item Utilizar correctamente "Scratch jr." para ejecutar soluciones previamente planteadas, tanto por el profesor, como él mismo.
			\item Crear historias y/o videojuegos mediante el uso de "Scratch jr." basados en temas de su interés.
			\item Fijar o adquirir conocimientos mediante la creación de historias y juegos con "Scratch jr.".
			\item Usar una computadora o móvil con normalidad.
		\end{itemize} &
		\begin{itemize}
			\item Comprende y analiza problemas sencillos de su entorno mediante el razonamiento lógico.
			\item Elabora y propone soluciones creativas a problemas sencillos de su entorno.
			\item Resuelve problemas mediante la ejecución de las soluciones propuestas, tanto por él mismo, como por el profesor.
			\item Reconoce diversos tipos de intrucciones y sus usos en el entorno de "Scratch jr.".
			\item Relaciona el "paso a paso" de una solución propuesta con instrucciones de "Scratch jr.".
			\item Asimila principios básicos de lógica (PNC, PI, PTE, PRS, PC, etc).
			\item Asimila estructuras lógicas simples (Conjunción, Disyunción inclusiva, Bicondicional, Condicional, etc).
			\item Entiende las conexiones entre causas y efectos y viceversa.
			\item Ordena ideas de manera coherente y cohesiva.
			\item Emplea diversos medios para la obtención de información (Preguntas al profesor, los compañeros o los padres, 
			uso de lógica deductiva, experimentación, etc).
		\end{itemize} &
		Diagramas ilustrativos, experimentos sencillos, presentaciones, "Scratch jr.", etc. &
		Ejercicios prácticos, Cuestionarios Orales y/o Escritos, Juegos, etc. &
		"¡Jugando también aprendo!"\par
		Este proyecto parte de la idea de que 'todo problema puede ser puesto bajo un modelo de datos e información'. 
		El mismo consiste en, a través de diversos juegos, hacer que los chicos puedan crear los suyos propios, estableciendo 
		reglas claras y sencillas, y tomando como temas alguno de su propio interés, de modo que en el proceso fije u obtenga 
		nuevos conocimientos acerca del mismo. \\
		\hline
		% Plan de Noviembre:
		Noviembre &
		Programación con "Scratch jr." &
		\begin{itemize}
			\item Analizar problemas simples de su entorno aplicando el razonamiento lógico.
			\item Resuelver problemas sencillos de su medio mediante el uso del razonamiento lógico.
			\item Mostrar curiosidad por las cosas de su entorno.
			\item Usar una computadora o móvil con normalidad.
		\end{itemize} &
		\begin{itemize}
			\item Comprende y analiza problemas sencillos de su entorno mediante el razonamiento lógico.
			\item Elabora y propone soluciones creativas a problemas sencillos de su entorno.
			\item Resuelve problemas mediante la ejecución de las soluciones propuestas, tanto por él mismo, como por el profesor.
			\item Emplea diversos medios para la obtención de información (Preguntas al profesor, los compañeros o los padres, 
			uso de lógica deductiva, experimentación, etc).
		\end{itemize} &
		Diagramas ilustrativos, experimentos sencillos, presentaciones, "Scratch jr.", etc.  &
		Ejercicios prácticos, Cuestionarios Orales y/o Escritos, Juegos, etc. &
		"¡Jugando también aprendo!"\par
		Este proyecto parte de la idea de que 'todo problema puede ser puesto bajo un modelo de datos e información'. 
		El mismo consiste en, a través de diversos juegos, hacer que los chicos puedan crear los suyos propios, estableciendo 
		reglas claras y sencillas, y tomando como temas alguno de su propio interés, de modo que en el proceso fije u obtenga 
		nuevos conocimientos acerca del mismo. \\
		\hline

	\end{longtable}
	\pagebreak[4]
	% ----------------------------------------------------------------------------------------------------------------------------------------
	% Cabecera:
	\begin{tabularx}{\textwidth}{ >{\raggedright\arraybackslash}X >{\centering\arraybackslash}X >{\raggedleft\arraybackslash}X }
		\includegraphics[width=0.3\textwidth]{\CEGSA} &
		\textbf{PLAN ANUAL DE CLASES} &
		\includegraphics[width=0.3\textwidth]{\CETSA}
	\end{tabularx}
	% Identificación:
	\begin{tabularx}{\textwidth}{ >{\raggedright\arraybackslash}X >{\raggedright\arraybackslash}X >{\raggedright\arraybackslash}X }
		Docente: \profesor &
		Turno: - &
		Año: \currentyear \\
		Disciplina: \discipline &
		Grado/Curso: 3er Grado &
		 \\
	\end{tabularx}
	% Contenido:
	\centering
	\begin{longtable}{|m{\smallcellwidth}|p{\normalcellwidth}|p{\bigcellwidth}|p{\bigcellwidth}|p{\normalcellwidth}|p{\normalcellwidth}|p{\normalcellwidth}|}
	%\begin{tabularx}{\textwidth}{|r|r|r|r|r|r|r|}
		\hline
		\textbf{Mes} &
		\textbf{Contenido/Unidad Temática} &
		\textbf{Capacidades} &
		\textbf{Indicadores} &
		\textbf{Recursos Didácticos/Uso de TIC's} &
		\textbf{Instrumentos de Evaluación} &
		\textbf{Proyectos Disciplinarios} \\
		\hline
		\endhead
		% Plan de Febrero:
		Febrero &
		Introducción a la Lógica &
		\begin{itemize}
			\item Hallar soluciones creativas a situaciones problemáticas mediante el uso del razonamiento lógico.
			\item Mostrar interés o curiosidad por las cosas de su entorno.
		\end{itemize} &
		\begin{itemize}
			\item Elabora estrategias para ganar diversos tipos de juegos de mesa (Casita robada, UNO, Ajedrez, Laberintos, etc).
			\item Resuelve diversos tipos de acertijos y puzzles correctamente.
			\item Emplea diversos medios para la obtención de información (Preguntas al profesor, los compañeros o los padres, uso de lógica deductiva, experimentación, etc).
 		\end{itemize} &
		Juegos diversos, Acertijos, Puzzles, Libros de cuentos, etc. &
		Questionarios Orales, Juegos o resolución de Acertijos y Trivias. &
		 - \\
		\hline
		% Plan de Marzo:
		Marzo &
		Introducción a la Programación con "Scratch" &
		\begin{itemize}
			\item Analizar problemas simples de su entorno aplicando el razonamiento lógico.
			\item Resuelver problemas sencillos de su medio mediante el uso del razonamiento lógico.
			\item Mostrar curiosidad por las cosas de su entorno.
			\item Usar una computadora o móvil con normalidad.
			\item Utilizar "Scratch" para la creación de programas sencillos.
			\item Utilizar "Scratch" para la ejecución de soluciones planteadas.
		\end{itemize} &
		\begin{itemize}
			\item Comprende y analiza problemas sencillos de su entorno mediante el razonamiento lógico.
			\item Elabora y propone soluciones creativas a problemas sencillos de su entorno.
			\item Resuelve problemas mediante la ejecución de las soluciones propuestas, tanto por él mismo, como por el profesor.
			\item Reconoce diversos tipos de intrucciones y sus usos en el entorno de "Scratch".
			\item Relaciona el "paso a paso" de una solución propuesta con instrucciones de "Scratch".
			\item Asimila principios básicos de lógica (PNC, PI, PTE, PRS, PC, etc).
			\item Asimila estructuras lógicas simples (Conjunción, Disyunción inclusiva, Bicondicional, Condicional, etc).
			\item Entiende las conexiones entre causas y efectos y viceversa.
			\item Ordena ideas de manera coherente y cohesiva.
			\item Emplea diversos medios para la obtención de información (Preguntas al profesor, los compañeros o los padres, 
			uso de lógica deductiva, experimentación, etc).
		\end{itemize} &
		Diagramas ilustrativos, experimentos sencillos, presentaciones, "Scratch", etc. &
		Ejercicios prácticos, Cuestionarios Orales y/o Escritos, Juegos, etc. &
		 \\
		\hline
		% Plan de Abril:
		Abril &
		 &
		\begin{itemize}
			\item 
		\end{itemize} &
		\begin{itemize}
			\item 
		\end{itemize} &
		  &
		  &
		 - \\
		\hline
		% Plan de Mayo:
		Mayo &
		 &
		\begin{itemize}
			\item 
		\end{itemize} &
		\begin{itemize}
			\item 
		\end{itemize} &
		  &
		  &
		 - \\
		\hline
		% Plan de Junio:
		Junio &
		 &
		\begin{itemize}
			\item 
		\end{itemize} &
		\begin{itemize}
			\item 
		\end{itemize} &
		  &
		  &
		 - \\
		\hline
		% Plan de Julio:
		Julio &
		 &
		\begin{itemize}
			\item 
		\end{itemize} &
		\begin{itemize}
			\item 
		\end{itemize} &
		  &
		  &
		 - \\
		\hline
		% Plan de Agosto:
		Agosto &
		 &
		\begin{itemize}
			\item 
		\end{itemize} &
		\begin{itemize}
			\item 
		\end{itemize} &
		  &
		  &
		 - \\
		\hline
		% Plan de Septiembre:
		Septiembre &
		 &
		\begin{itemize}
			\item 
		\end{itemize} &
		\begin{itemize}
			\item 
		\end{itemize} &
		  &
		  &
		 - \\
		\hline
		% Plan de Octubre:
		Octubre &
		 &
		\begin{itemize}
			\item 
		\end{itemize} &
		\begin{itemize}
			\item 
		\end{itemize} &
		  &
		  &
		 - \\
		\hline
		% Plan de Noviembre:
		Noviembre &
		 &
		\begin{itemize}
			\item 
		\end{itemize} &
		\begin{itemize}
			\item 
		\end{itemize} &
		  &
		  &
		 - \\
		\hline

	\end{longtable}
	\pagebreak[4]
	% ----------------------------------------------------------------------------------------------------------------------------------------
	% Cabecera:
	\begin{tabularx}{\textwidth}{ >{\raggedright\arraybackslash}X >{\centering\arraybackslash}X >{\raggedleft\arraybackslash}X }
		\includegraphics[width=0.3\textwidth]{\CEGSA} &
		\textbf{PLAN ANUAL DE CLASES} &
		\includegraphics[width=0.3\textwidth]{\CETSA}
	\end{tabularx}
	% Identificación:
	\begin{tabularx}{\textwidth}{ >{\raggedright\arraybackslash}X >{\raggedright\arraybackslash}X >{\raggedright\arraybackslash}X }
		Docente: \profesor &
		Turno: - &
		Año: \currentyear \\
		Disciplina: \discipline &
		Grado/Curso: 4to Grado &
		 \\
	\end{tabularx}
	% Contenido:
	\centering
	\begin{longtable}{|m{\smallcellwidth}|p{\normalcellwidth}|p{\bigcellwidth}|p{\bigcellwidth}|p{\normalcellwidth}|p{\normalcellwidth}|p{\normalcellwidth}|}
	%\begin{tabularx}{\textwidth}{|r|r|r|r|r|r|r|}
		\hline
		\textbf{Mes} &
		\textbf{Contenido/Unidad Temática} &
		\textbf{Capacidades} &
		\textbf{Indicadores} &
		\textbf{Recursos Didácticos/Uso de TIC's} &
		\textbf{Instrumentos de Evaluación} &
		\textbf{Proyectos Disciplinarios} \\
		\hline
		\endhead
		% Plan de Febrero:
		Febrero &
		Introducción a la Lógica &
		\begin{itemize}
			\item Hallar soluciones creativas a situaciones problemáticas mediante el uso del razonamiento lógico.
			\item Mostrar interés o curiosidad por las cosas de su entorno.
		\end{itemize} &
		\begin{itemize}
			\item Elabora estrategias para ganar diversos tipos de juegos de mesa (Casita robada, UNO, Ajedrez, Laberintos, etc).
			\item Resuelve diversos tipos de acertijos y puzzles correctamente.
			\item Emplea diversos medios para la obtención de información (Preguntas al profesor, los compañeros o los padres, uso de lógica deductiva, experimentación, etc).
 		\end{itemize} &
		Juegos diversos, Acertijos, Puzzles, Libros de cuentos, etc. &
		Questionarios Orales, Juegos o resolución de Acertijos y Trivias. &
		 - \\
		\hline
		% Plan de Marzo:
		Marzo &
		 &
		\begin{itemize}
			\item 
		\end{itemize} &
		\begin{itemize}
			\item 
		\end{itemize} &
		  &
		  &
		 - \\
		\hline
		% Plan de Abril:
		Abril &
		 &
		\begin{itemize}
			\item 
		\end{itemize} &
		\begin{itemize}
			\item 
		\end{itemize} &
		  &
		  &
		 - \\
		\hline
		% Plan de Mayo:
		Mayo &
		 &
		\begin{itemize}
			\item 
		\end{itemize} &
		\begin{itemize}
			\item 
		\end{itemize} &
		  &
		  &
		 - \\
		\hline
		% Plan de Junio:
		Junio &
		 &
		\begin{itemize}
			\item 
		\end{itemize} &
		\begin{itemize}
			\item 
		\end{itemize} &
		  &
		  &
		 - \\
		\hline
		% Plan de Julio:
		Julio &
		 &
		\begin{itemize}
			\item 
		\end{itemize} &
		\begin{itemize}
			\item 
		\end{itemize} &
		  &
		  &
		 - \\
		\hline
		% Plan de Agosto:
		Agosto &
		 &
		\begin{itemize}
			\item 
		\end{itemize} &
		\begin{itemize}
			\item 
		\end{itemize} &
		  &
		  &
		 - \\
		\hline
		% Plan de Septiembre:
		Septiembre &
		 &
		\begin{itemize}
			\item 
		\end{itemize} &
		\begin{itemize}
			\item 
		\end{itemize} &
		  &
		  &
		 - \\
		\hline
		% Plan de Octubre:
		Octubre &
		 &
		\begin{itemize}
			\item 
		\end{itemize} &
		\begin{itemize}
			\item 
		\end{itemize} &
		  &
		  &
		 - \\
		\hline
		% Plan de Noviembre:
		Noviembre &
		 &
		\begin{itemize}
			\item 
		\end{itemize} &
		\begin{itemize}
			\item 
		\end{itemize} &
		  &
		  &
		 - \\
		\hline

	\end{longtable}
	\pagebreak[4]
	% ----------------------------------------------------------------------------------------------------------------------------------------
	% Cabecera:
	\begin{tabularx}{\textwidth}{ >{\raggedright\arraybackslash}X >{\centering\arraybackslash}X >{\raggedleft\arraybackslash}X }
		\includegraphics[width=0.3\textwidth]{\CEGSA} &
		\textbf{PLAN ANUAL DE CLASES} &
		\includegraphics[width=0.3\textwidth]{\CETSA}
	\end{tabularx}
	% Identificación:
	\begin{tabularx}{\textwidth}{ >{\raggedright\arraybackslash}X >{\raggedright\arraybackslash}X >{\raggedright\arraybackslash}X }
		Docente: \profesor &
		Turno: - &
		Año: \currentyear \\
		Disciplina: \discipline &
		Grado/Curso: 5to Grado &
		 \\
	\end{tabularx}
	% Contenido:
	\centering
	\begin{longtable}{|m{\smallcellwidth}|p{\normalcellwidth}|p{\bigcellwidth}|p{\bigcellwidth}|p{\normalcellwidth}|p{\normalcellwidth}|p{\normalcellwidth}|}
	%\begin{tabularx}{\textwidth}{|r|r|r|r|r|r|r|}
		\hline
		\textbf{Mes} &
		\textbf{Contenido/Unidad Temática} &
		\textbf{Capacidades} &
		\textbf{Indicadores} &
		\textbf{Recursos Didácticos/Uso de TIC's} &
		\textbf{Instrumentos de Evaluación} &
		\textbf{Proyectos Disciplinarios} \\
		\hline
		\endhead
		% Plan de Febrero:
		Febrero &
		Introducción a la Lógica &
		\begin{itemize}
			\item Hallar soluciones creativas a situaciones problemáticas mediante el uso del razonamiento lógico.
			\item Mostrar interés o curiosidad por las cosas de su entorno.
		\end{itemize} &
		\begin{itemize}
			\item Elabora estrategias para ganar diversos tipos de juegos de mesa (Casita robada, UNO, Ajedrez, Laberintos, etc).
			\item Resuelve diversos tipos de acertijos y puzzles correctamente.
			\item Emplea diversos medios para la obtención de información (Preguntas al profesor, los compañeros o los padres, uso de lógica deductiva, experimentación, etc).
 		\end{itemize} &
		Juegos diversos, Acertijos, Puzzles, Libros de cuentos, etc. &
		Questionarios Orales, Juegos o resolución de Acertijos y Trivias. &
		 - \\
		\hline
		% Plan de Marzo:
		Marzo &
		 &
		\begin{itemize}
			\item 
		\end{itemize} &
		\begin{itemize}
			\item 
		\end{itemize} &
		  &
		  &
		 - \\
		\hline
		% Plan de Abril:
		Abril &
		 &
		\begin{itemize}
			\item 
		\end{itemize} &
		\begin{itemize}
			\item 
		\end{itemize} &
		  &
		  &
		 - \\
		\hline
		% Plan de Mayo:
		Mayo &
		 &
		\begin{itemize}
			\item 
		\end{itemize} &
		\begin{itemize}
			\item 
		\end{itemize} &
		  &
		  &
		 - \\
		\hline
		% Plan de Junio:
		Junio &
		 &
		\begin{itemize}
			\item 
		\end{itemize} &
		\begin{itemize}
			\item 
		\end{itemize} &
		  &
		  &
		 - \\
		\hline
		% Plan de Julio:
		Julio &
		 &
		\begin{itemize}
			\item 
		\end{itemize} &
		\begin{itemize}
			\item 
		\end{itemize} &
		  &
		  &
		 - \\
		\hline
		% Plan de Agosto:
		Agosto &
		 &
		\begin{itemize}
			\item 
		\end{itemize} &
		\begin{itemize}
			\item 
		\end{itemize} &
		  &
		  &
		 - \\
		\hline
		% Plan de Septiembre:
		Septiembre &
		 &
		\begin{itemize}
			\item 
		\end{itemize} &
		\begin{itemize}
			\item 
		\end{itemize} &
		  &
		  &
		 - \\
		\hline
		% Plan de Octubre:
		Octubre &
		 &
		\begin{itemize}
			\item 
		\end{itemize} &
		\begin{itemize}
			\item 
		\end{itemize} &
		  &
		  &
		 - \\
		\hline
		% Plan de Noviembre:
		Noviembre &
		 &
		\begin{itemize}
			\item 
		\end{itemize} &
		\begin{itemize}
			\item 
		\end{itemize} &
		  &
		  &
		 - \\
		\hline

	\end{longtable}
	\pagebreak[4]
	% ----------------------------------------------------------------------------------------------------------------------------------------
	% Cabecera:
	\begin{tabularx}{\textwidth}{ >{\raggedright\arraybackslash}X >{\centering\arraybackslash}X >{\raggedleft\arraybackslash}X }
		\includegraphics[width=0.3\textwidth]{\CEGSA} &
		\textbf{PLAN ANUAL DE CLASES} &
		\includegraphics[width=0.3\textwidth]{\CETSA}
	\end{tabularx}
	% Identificación:
	\begin{tabularx}{\textwidth}{ >{\raggedright\arraybackslash}X >{\raggedright\arraybackslash}X >{\raggedright\arraybackslash}X }
		Docente: \profesor &
		Turno: - &
		Año: \currentyear \\
		Disciplina: \discipline &
		Grado/Curso: 6to Grado &
		 \\
	\end{tabularx}
	% Contenido:
	\centering
	\begin{longtable}{|m{\smallcellwidth}|p{\normalcellwidth}|p{\bigcellwidth}|p{\bigcellwidth}|p{\normalcellwidth}|p{\normalcellwidth}|p{\normalcellwidth}|}
	%\begin{tabularx}{\textwidth}{|r|r|r|r|r|r|r|}
		\hline
		\textbf{Mes} &
		\textbf{Contenido/Unidad Temática} &
		\textbf{Capacidades} &
		\textbf{Indicadores} &
		\textbf{Recursos Didácticos/Uso de TIC's} &
		\textbf{Instrumentos de Evaluación} &
		\textbf{Proyectos Disciplinarios} \\
		\hline
		\endhead
		% Plan de Febrero:
		Febrero &
		Introducción a la Lógica &
		\begin{itemize}
			\item Hallar soluciones creativas a situaciones problemáticas mediante el uso del razonamiento lógico.
			\item Mostrar interés o curiosidad por las cosas de su entorno.
		\end{itemize} &
		\begin{itemize}
			\item Elabora estrategias para ganar diversos tipos de juegos de mesa (Casita robada, UNO, Ajedrez, Laberintos, etc).
			\item Resuelve diversos tipos de acertijos y puzzles correctamente.
			\item Emplea diversos medios para la obtención de información (Preguntas al profesor, los compañeros o los padres, uso de lógica deductiva, experimentación, etc).
 		\end{itemize} &
		Juegos diversos, Acertijos, Puzzles, Libros de cuentos, etc. &
		Questionarios Orales, Juegos o resolución de Acertijos y Trivias. &
		 - \\
		\hline
		% Plan de Marzo:
		Marzo &
		 &
		\begin{itemize}
			\item 
		\end{itemize} &
		\begin{itemize}
			\item 
		\end{itemize} &
		  &
		  &
		 - \\
		\hline
		% Plan de Abril:
		Abril &
		 &
		\begin{itemize}
			\item 
		\end{itemize} &
		\begin{itemize}
			\item 
		\end{itemize} &
		  &
		  &
		 - \\
		\hline
		% Plan de Mayo:
		Mayo &
		 &
		\begin{itemize}
			\item 
		\end{itemize} &
		\begin{itemize}
			\item 
		\end{itemize} &
		  &
		  &
		 - \\
		\hline
		% Plan de Junio:
		Junio &
		 &
		\begin{itemize}
			\item 
		\end{itemize} &
		\begin{itemize}
			\item 
		\end{itemize} &
		  &
		  &
		 - \\
		\hline
		% Plan de Julio:
		Julio &
		 &
		\begin{itemize}
			\item 
		\end{itemize} &
		\begin{itemize}
			\item 
		\end{itemize} &
		  &
		  &
		 - \\
		\hline
		% Plan de Agosto:
		Agosto &
		 &
		\begin{itemize}
			\item 
		\end{itemize} &
		\begin{itemize}
			\item 
		\end{itemize} &
		  &
		  &
		 - \\
		\hline
		% Plan de Septiembre:
		Septiembre &
		 &
		\begin{itemize}
			\item 
		\end{itemize} &
		\begin{itemize}
			\item 
		\end{itemize} &
		  &
		  &
		 - \\
		\hline
		% Plan de Octubre:
		Octubre &
		 &
		\begin{itemize}
			\item 
		\end{itemize} &
		\begin{itemize}
			\item 
		\end{itemize} &
		  &
		  &
		 - \\
		\hline
		% Plan de Noviembre:
		Noviembre &
		 &
		\begin{itemize}
			\item 
		\end{itemize} &
		\begin{itemize}
			\item 
		\end{itemize} &
		  &
		  &
		 - \\
		\hline

	\end{longtable}
	\pagebreak[4]
	% ----------------------------------------------------------------------------------------------------------------------------------------
	% Cabecera:
	\begin{tabularx}{\textwidth}{ >{\raggedright\arraybackslash}X >{\centering\arraybackslash}X >{\raggedleft\arraybackslash}X }
		\includegraphics[width=0.3\textwidth]{\CEGSA} &
		\textbf{PLAN ANUAL DE CLASES} &
		\includegraphics[width=0.3\textwidth]{\CETSA}
	\end{tabularx}
	% Identificación:
	\begin{tabularx}{\textwidth}{ >{\raggedright\arraybackslash}X >{\raggedright\arraybackslash}X >{\raggedright\arraybackslash}X }
		Docente: \profesor &
		Turno: - &
		Año: \currentyear \\
		Disciplina: \discipline &
		Grado/Curso: 7mo Grado &
		 \\
	\end{tabularx}
	% Contenido:
	\centering
	\begin{longtable}{|m{\smallcellwidth}|p{\normalcellwidth}|p{\bigcellwidth}|p{\bigcellwidth}|p{\normalcellwidth}|p{\normalcellwidth}|p{\normalcellwidth}|}
	
	%\begin{tabularx}{\textwidth}{|r|r|r|r|r|r|r|}
		\hline
		\textbf{Mes} &
		\textbf{Contenido/Unidad Temática} &
		\textbf{Capacidades} &
		\textbf{Indicadores} &
		\textbf{Recursos Didácticos/Uso de TIC's} &
		\textbf{Instrumentos de Evaluación} &
		\textbf{Proyectos Disciplinarios} \\
		\hline
		\endhead
		% Plan de Febrero:
		Febrero &
		Introducción a la Lógica &
		\begin{itemize}
			\item Hallar soluciones creativas a situaciones problemáticas mediante el uso del razonamiento lógico.
			\item Mostrar interés o curiosidad por las cosas de su entorno.
		\end{itemize} &
		\begin{itemize}
			\item Elabora estrategias para ganar diversos tipos de juegos de mesa (Casita robada, UNO, Ajedrez, Laberintos, etc).
			\item Resuelve diversos tipos de acertijos y puzzles correctamente.
			\item Emplea diversos medios para la obtención de información (Preguntas al profesor, los compañeros o los padres, uso de lógica deductiva, experimentación, etc).
 		\end{itemize} &
		Juegos diversos, Acertijos, Puzzles, Libros de cuentos, etc. &
		Questionarios Orales, Juegos o resolución de Acertijos y Trivias. &
		 - \\
		\hline
		% Plan de Marzo:
		Marzo &
		 &
		\begin{itemize}
			\item 
		\end{itemize} &
		\begin{itemize}
			\item 
		\end{itemize} &
		  &
		  &
		 - \\
		\hline
		% Plan de Abril:
		Abril &
		 &
		\begin{itemize}
			\item 
		\end{itemize} &
		\begin{itemize}
			\item 
		\end{itemize} &
		  &
		  &
		 - \\
		\hline
		% Plan de Mayo:
		Mayo &
		 &
		\begin{itemize}
			\item 
		\end{itemize} &
		\begin{itemize}
			\item 
		\end{itemize} &
		  &
		  &
		 - \\
		\hline
		% Plan de Junio:
		Junio &
		 &
		\begin{itemize}
			\item 
		\end{itemize} &
		\begin{itemize}
			\item 
		\end{itemize} &
		  &
		  &
		 - \\
		\hline
		% Plan de Julio:
		Julio &
		 &
		\begin{itemize}
			\item 
		\end{itemize} &
		\begin{itemize}
			\item 
		\end{itemize} &
		  &
		  &
		 - \\
		\hline
		% Plan de Agosto:
		Agosto &
		 &
		\begin{itemize}
			\item 
		\end{itemize} &
		\begin{itemize}
			\item 
		\end{itemize} &
		  &
		  &
		 - \\
		\hline
		% Plan de Septiembre:
		Septiembre &
		 &
		\begin{itemize}
			\item 
		\end{itemize} &
		\begin{itemize}
			\item 
		\end{itemize} &
		  &
		  &
		 - \\
		\hline
		% Plan de Octubre:
		Octubre &
		 &
		\begin{itemize}
			\item 
		\end{itemize} &
		\begin{itemize}
			\item 
		\end{itemize} &
		  &
		  &
		 - \\
		\hline
		% Plan de Noviembre:
		Noviembre &
		 &
		\begin{itemize}
			\item 
		\end{itemize} &
		\begin{itemize}
			\item 
		\end{itemize} &
		  &
		  &
		 - \\
		\hline

	\end{longtable}
	\pagebreak[4]
	% ----------------------------------------------------------------------------------------------------------------------------------------
	% Cabecera:
	\begin{tabularx}{\textwidth}{ >{\raggedright\arraybackslash}X >{\centering\arraybackslash}X >{\raggedleft\arraybackslash}X }
		\includegraphics[width=0.3\textwidth]{\CEGSA} &
		\textbf{PLAN ANUAL DE CLASES} &
		\includegraphics[width=0.3\textwidth]{\CETSA}
	\end{tabularx}
	% Identificación:
	\begin{tabularx}{\textwidth}{ >{\raggedright\arraybackslash}X >{\raggedright\arraybackslash}X >{\raggedright\arraybackslash}X }
		Docente: \profesor &
		Turno: - &
		Año: \currentyear \\
		Disciplina: \discipline &
		Grado/Curso: 8vo Grado &
		 \\
	\end{tabularx}
	% Contenido:
	\centering
	\begin{longtable}{|m{\smallcellwidth}|p{\normalcellwidth}|p{\bigcellwidth}|p{\bigcellwidth}|p{\normalcellwidth}|p{\normalcellwidth}|p{\normalcellwidth}|}
	%\begin{tabularx}{\textwidth}{|r|r|r|r|r|r|r|}
		\hline
		\textbf{Mes} &
		\textbf{Contenido/Unidad Temática} &
		\textbf{Capacidades} &
		\textbf{Indicadores} &
		\textbf{Recursos Didácticos/Uso de TIC's} &
		\textbf{Instrumentos de Evaluación} &
		\textbf{Proyectos Disciplinarios} \\
		\hline
		\endhead
		% Plan de Febrero:
		Febrero &
		Introducción a la Lógica &
		\begin{itemize}
			\item Hallar soluciones creativas a situaciones problemáticas mediante el uso del razonamiento lógico.
			\item Mostrar interés o curiosidad por las cosas de su entorno.
		\end{itemize} &
		\begin{itemize}
			\item Elabora estrategias para ganar diversos tipos de juegos de mesa (Casita robada, UNO, Ajedrez, Laberintos, etc).
			\item Resuelve diversos tipos de acertijos y puzzles correctamente.
			\item Emplea diversos medios para la obtención de información (Preguntas al profesor, los compañeros o los padres, uso de lógica deductiva, experimentación, etc).
 		\end{itemize} &
		Juegos diversos, Acertijos, Puzzles, Libros de cuentos, etc. &
		Questionarios Orales, Juegos o resolución de Acertijos y Trivias. &
		 - \\
		\hline
		% Plan de Marzo:
		Marzo &
		 &
		\begin{itemize}
			\item 
		\end{itemize} &
		\begin{itemize}
			\item 
		\end{itemize} &
		  &
		  &
		 - \\
		\hline
		% Plan de Abril:
		Abril &
		 &
		\begin{itemize}
			\item 
		\end{itemize} &
		\begin{itemize}
			\item 
		\end{itemize} &
		  &
		  &
		 - \\
		\hline
		% Plan de Mayo:
		Mayo &
		 &
		\begin{itemize}
			\item 
		\end{itemize} &
		\begin{itemize}
			\item 
		\end{itemize} &
		  &
		  &
		 - \\
		\hline
		% Plan de Junio:
		Junio &
		 &
		\begin{itemize}
			\item 
		\end{itemize} &
		\begin{itemize}
			\item 
		\end{itemize} &
		  &
		  &
		 - \\
		\hline
		% Plan de Julio:
		Julio &
		 &
		\begin{itemize}
			\item 
		\end{itemize} &
		\begin{itemize}
			\item 
		\end{itemize} &
		  &
		  &
		 - \\
		\hline
		% Plan de Agosto:
		Agosto &
		 &
		\begin{itemize}
			\item 
		\end{itemize} &
		\begin{itemize}
			\item 
		\end{itemize} &
		  &
		  &
		 - \\
		\hline
		% Plan de Septiembre:
		Septiembre &
		 &
		\begin{itemize}
			\item 
		\end{itemize} &
		\begin{itemize}
			\item 
		\end{itemize} &
		  &
		  &
		 - \\
		\hline
		% Plan de Octubre:
		Octubre &
		 &
		\begin{itemize}
			\item 
		\end{itemize} &
		\begin{itemize}
			\item 
		\end{itemize} &
		  &
		  &
		 - \\
		\hline
		% Plan de Noviembre:
		Noviembre &
		 &
		\begin{itemize}
			\item 
		\end{itemize} &
		\begin{itemize}
			\item 
		\end{itemize} &
		  &
		  &
		 - \\
		\hline

	\end{longtable}
	\pagebreak[4]
	% ----------------------------------------------------------------------------------------------------------------------------------------
	% Cabecera:
	\begin{tabularx}{\textwidth}{ >{\raggedright\arraybackslash}X >{\centering\arraybackslash}X >{\raggedleft\arraybackslash}X }
		\includegraphics[width=0.3\textwidth]{\CEGSA} &
		\textbf{PLAN ANUAL DE CLASES} &
		\includegraphics[width=0.3\textwidth]{\CETSA}
	\end{tabularx}
	% Identificación:
	\begin{tabularx}{\textwidth}{ >{\raggedright\arraybackslash}X >{\raggedright\arraybackslash}X >{\raggedright\arraybackslash}X }
		Docente: \profesor &
		Turno: - &
		Año: \currentyear \\
		Disciplina: \discipline &
		Grado/Curso: 9no Grado &
		 \\
	\end{tabularx}
	% Contenido:
	\centering
	\begin{longtable}{|m{\smallcellwidth}|p{\normalcellwidth}|p{\bigcellwidth}|p{\bigcellwidth}|p{\normalcellwidth}|p{\normalcellwidth}|p{\normalcellwidth}|}
	%\begin{tabularx}{\textwidth}{|r|r|r|r|r|r|r|}
		\hline
		\textbf{Mes} &
		\textbf{Contenido/Unidad Temática} &
		\textbf{Capacidades} &
		\textbf{Indicadores} &
		\textbf{Recursos Didácticos/Uso de TIC's} &
		\textbf{Instrumentos de Evaluación} &
		\textbf{Proyectos Disciplinarios} \\
		\hline
		\endhead
		% Plan de Febrero:
		Febrero &
		Introducción a la Lógica &
		\begin{itemize}
			\item Hallar soluciones creativas a situaciones problemáticas mediante el uso del razonamiento lógico.
			\item Mostrar interés o curiosidad por las cosas de su entorno.
		\end{itemize} &
		\begin{itemize}
			\item Elabora estrategias para ganar diversos tipos de juegos de mesa (Casita robada, UNO, Ajedrez, Laberintos, etc).
			\item Resuelve diversos tipos de acertijos y puzzles correctamente.
			\item Emplea diversos medios para la obtención de información (Preguntas al profesor, los compañeros o los padres, uso de lógica deductiva, experimentación, etc).
 		\end{itemize} &
		Juegos diversos, Acertijos, Puzzles, Libros de cuentos, etc. &
		Questionarios Orales, Juegos o resolución de Acertijos y Trivias. &
		 - \\
		\hline
		% Plan de Marzo:
		Marzo &
		 &
		\begin{itemize}
			\item 
		\end{itemize} &
		\begin{itemize}
			\item 
		\end{itemize} &
		  &
		  &
		 - \\
		\hline
		% Plan de Abril:
		Abril &
		 &
		\begin{itemize}
			\item 
		\end{itemize} &
		\begin{itemize}
			\item 
		\end{itemize} &
		  &
		  &
		 - \\
		\hline
		% Plan de Mayo:
		Mayo &
		 &
		\begin{itemize}
			\item 
		\end{itemize} &
		\begin{itemize}
			\item 
		\end{itemize} &
		  &
		  &
		 - \\
		\hline
		% Plan de Junio:
		Junio &
		 &
		\begin{itemize}
			\item 
		\end{itemize} &
		\begin{itemize}
			\item 
		\end{itemize} &
		  &
		  &
		 - \\
		\hline
		% Plan de Julio:
		Julio &
		 &
		\begin{itemize}
			\item 
		\end{itemize} &
		\begin{itemize}
			\item 
		\end{itemize} &
		  &
		  &
		 - \\
		\hline
		% Plan de Agosto:
		Agosto &
		 &
		\begin{itemize}
			\item 
		\end{itemize} &
		\begin{itemize}
			\item 
		\end{itemize} &
		  &
		  &
		 - \\
		\hline
		% Plan de Septiembre:
		Septiembre &
		 &
		\begin{itemize}
			\item 
		\end{itemize} &
		\begin{itemize}
			\item 
		\end{itemize} &
		  &
		  &
		 - \\
		\hline
		% Plan de Octubre:
		Octubre &
		 &
		\begin{itemize}
			\item 
		\end{itemize} &
		\begin{itemize}
			\item 
		\end{itemize} &
		  &
		  &
		 - \\
		\hline
		% Plan de Noviembre:
		Noviembre &
		 &
		\begin{itemize}
			\item 
		\end{itemize} &
		\begin{itemize}
			\item 
		\end{itemize} &
		  &
		  &
		 - \\
		\hline

	\end{longtable}
	\pagebreak[4]
	% ----------------------------------------------------------------------------------------------------------------------------------------
	% Cabecera:
	\begin{tabularx}{\textwidth}{ >{\raggedright\arraybackslash}X >{\centering\arraybackslash}X >{\raggedleft\arraybackslash}X }
		\includegraphics[width=0.3\textwidth]{\CEGSA} &
		\textbf{PLAN ANUAL DE CLASES} &
		\includegraphics[width=0.3\textwidth]{\CETSA}
	\end{tabularx}
	% Identificación:
	\begin{tabularx}{\textwidth}{ >{\raggedright\arraybackslash}X >{\raggedright\arraybackslash}X >{\raggedright\arraybackslash}X }
		Docente: \profesor &
		Turno: - &
		Año: \currentyear \\
		Disciplina: \discipline &
		Grado/Curso: 1er Curso &
		 \\
	\end{tabularx}
	% Contenido:
	\centering
	\begin{longtable}{|m{\smallcellwidth}|p{\normalcellwidth}|p{\bigcellwidth}|p{\bigcellwidth}|p{\normalcellwidth}|p{\normalcellwidth}|p{\normalcellwidth}|}
	%\begin{tabularx}{\textwidth}{|r|r|r|r|r|r|r|}
		\hline
		\textbf{Mes} &
		\textbf{Contenido/Unidad Temática} &
		\textbf{Capacidades} &
		\textbf{Indicadores} &
		\textbf{Recursos Didácticos/Uso de TIC's} &
		\textbf{Instrumentos de Evaluación} &
		\textbf{Proyectos Disciplinarios} \\
		\hline
		\endhead
		% Plan de Febrero:
		Febrero &
		Introducción a la Lógica &
		\begin{itemize}
			\item Hallar soluciones creativas a situaciones problemáticas mediante el uso del razonamiento lógico.
			\item Mostrar interés o curiosidad por las cosas de su entorno.
		\end{itemize} &
		\begin{itemize}
			\item Elabora estrategias para ganar diversos tipos de juegos de mesa (Casita robada, UNO, Ajedrez, Laberintos, etc).
			\item Resuelve diversos tipos de acertijos y puzzles correctamente.
			\item Emplea diversos medios para la obtención de información (Preguntas al profesor, los compañeros o los padres, uso de lógica deductiva, experimentación, etc).
 		\end{itemize} &
		Juegos diversos, Acertijos, Puzzles, Libros de cuentos, etc. &
		Questionarios Orales, Juegos o resolución de Acertijos y Trivias. &
		 - \\
		\hline
		% Plan de Marzo:
		Marzo &
		 &
		\begin{itemize}
			\item 
		\end{itemize} &
		\begin{itemize}
			\item 
		\end{itemize} &
		  &
		  &
		 - \\
		\hline
		% Plan de Abril:
		Abril &
		 &
		\begin{itemize}
			\item 
		\end{itemize} &
		\begin{itemize}
			\item 
		\end{itemize} &
		  &
		  &
		 - \\
		\hline
		% Plan de Mayo:
		Mayo &
		 &
		\begin{itemize}
			\item 
		\end{itemize} &
		\begin{itemize}
			\item 
		\end{itemize} &
		  &
		  &
		 - \\
		\hline
		% Plan de Junio:
		Junio &
		 &
		\begin{itemize}
			\item 
		\end{itemize} &
		\begin{itemize}
			\item 
		\end{itemize} &
		  &
		  &
		 - \\
		\hline
		% Plan de Julio:
		Julio &
		 &
		\begin{itemize}
			\item 
		\end{itemize} &
		\begin{itemize}
			\item 
		\end{itemize} &
		  &
		  &
		 - \\
		\hline
		% Plan de Agosto:
		Agosto &
		 &
		\begin{itemize}
			\item 
		\end{itemize} &
		\begin{itemize}
			\item 
		\end{itemize} &
		  &
		  &
		 - \\
		\hline
		% Plan de Septiembre:
		Septiembre &
		 &
		\begin{itemize}
			\item 
		\end{itemize} &
		\begin{itemize}
			\item 
		\end{itemize} &
		  &
		  &
		 - \\
		\hline
		% Plan de Octubre:
		Octubre &
		 &
		\begin{itemize}
			\item 
		\end{itemize} &
		\begin{itemize}
			\item 
		\end{itemize} &
		  &
		  &
		 - \\
		\hline
		% Plan de Noviembre:
		Noviembre &
		 &
		\begin{itemize}
			\item 
		\end{itemize} &
		\begin{itemize}
			\item 
		\end{itemize} &
		  &
		  &
		 - \\
		\hline

	\end{longtable}
	\pagebreak[4]
	% ----------------------------------------------------------------------------------------------------------------------------------------
	% Cabecera:
	\begin{tabularx}{\textwidth}{ >{\raggedright\arraybackslash}X >{\centering\arraybackslash}X >{\raggedleft\arraybackslash}X }
		\includegraphics[width=0.3\textwidth]{\CEGSA} &
		\textbf{PLAN ANUAL DE CLASES} &
		\includegraphics[width=0.3\textwidth]{\CETSA}
	\end{tabularx}
	% Identificación:
	\begin{tabularx}{\textwidth}{ >{\raggedright\arraybackslash}X >{\raggedright\arraybackslash}X >{\raggedright\arraybackslash}X }
		Docente: \profesor &
		Turno: - &
		Año: \currentyear \\
		Disciplina: \discipline &
		Grado/Curso: 2do Curso &
		 \\
	\end{tabularx}
	% Contenido:
	\centering
	\begin{longtable}{|m{\smallcellwidth}|p{\normalcellwidth}|p{\bigcellwidth}|p{\bigcellwidth}|p{\normalcellwidth}|p{\normalcellwidth}|p{\normalcellwidth}|}
	%\begin{tabularx}{\textwidth}{|r|r|r|r|r|r|r|}
		\hline
		\textbf{Mes} &
		\textbf{Contenido/Unidad Temática} &
		\textbf{Capacidades} &
		\textbf{Indicadores} &
		\textbf{Recursos Didácticos/Uso de TIC's} &
		\textbf{Instrumentos de Evaluación} &
		\textbf{Proyectos Disciplinarios} \\
		\hline
		\endhead
		% Plan de Febrero:
		Febrero &
		Introducción a la Lógica &
		\begin{itemize}
			\item Hallar soluciones creativas a situaciones problemáticas mediante el uso del razonamiento lógico.
			\item Mostrar interés o curiosidad por las cosas de su entorno.
		\end{itemize} &
		\begin{itemize}
			\item Elabora estrategias para ganar diversos tipos de juegos de mesa (Casita robada, UNO, Ajedrez, Laberintos, etc).
			\item Resuelve diversos tipos de acertijos y puzzles correctamente.
			\item Emplea diversos medios para la obtención de información (Preguntas al profesor, los compañeros o los padres, uso de lógica deductiva, experimentación, etc).
 		\end{itemize} &
		Juegos diversos, Acertijos, Puzzles, Libros de cuentos, etc. &
		Questionarios Orales, Juegos o resolución de Acertijos y Trivias. &
		 - \\
		\hline
		% Plan de Marzo:
		Marzo &
		 &
		\begin{itemize}
			\item 
		\end{itemize} &
		\begin{itemize}
			\item 
		\end{itemize} &
		  &
		  &
		 - \\
		\hline
		% Plan de Abril:
		Abril &
		 &
		\begin{itemize}
			\item 
		\end{itemize} &
		\begin{itemize}
			\item 
		\end{itemize} &
		  &
		  &
		 - \\
		\hline
		% Plan de Mayo:
		Mayo &
		 &
		\begin{itemize}
			\item 
		\end{itemize} &
		\begin{itemize}
			\item 
		\end{itemize} &
		  &
		  &
		 - \\
		\hline
		% Plan de Junio:
		Junio &
		 &
		\begin{itemize}
			\item 
		\end{itemize} &
		\begin{itemize}
			\item 
		\end{itemize} &
		  &
		  &
		 - \\
		\hline
		% Plan de Julio:
		Julio &
		 &
		\begin{itemize}
			\item 
		\end{itemize} &
		\begin{itemize}
			\item 
		\end{itemize} &
		  &
		  &
		 - \\
		\hline
		% Plan de Agosto:
		Agosto &
		 &
		\begin{itemize}
			\item 
		\end{itemize} &
		\begin{itemize}
			\item 
		\end{itemize} &
		  &
		  &
		 - \\
		\hline
		% Plan de Septiembre:
		Septiembre &
		 &
		\begin{itemize}
			\item 
		\end{itemize} &
		\begin{itemize}
			\item 
		\end{itemize} &
		  &
		  &
		 - \\
		\hline
		% Plan de Octubre:
		Octubre &
		 &
		\begin{itemize}
			\item 
		\end{itemize} &
		\begin{itemize}
			\item 
		\end{itemize} &
		  &
		  &
		 - \\
		\hline
		% Plan de Noviembre:
		Noviembre &
		 &
		\begin{itemize}
			\item 
		\end{itemize} &
		\begin{itemize}
			\item 
		\end{itemize} &
		  &
		  &
		 - \\
		\hline

	\end{longtable}
	\pagebreak[4]
	% ----------------------------------------------------------------------------------------------------------------------------------------
	% Cabecera:
	\begin{tabularx}{\textwidth}{ >{\raggedright\arraybackslash}X >{\centering\arraybackslash}X >{\raggedleft\arraybackslash}X }
		\includegraphics[width=0.3\textwidth]{\CEGSA} &
		\textbf{PLAN ANUAL DE CLASES} &
		\includegraphics[width=0.3\textwidth]{\CETSA}
	\end{tabularx}
	% Identificación:
	\begin{tabularx}{\textwidth}{ >{\raggedright\arraybackslash}X >{\raggedright\arraybackslash}X >{\raggedright\arraybackslash}X }
		Docente: \profesor &
		Turno: - &
		Año: \currentyear \\
		Disciplina: \discipline &
		Grado/Curso: 3er Curso &
		 \\
	\end{tabularx}
	% Contenido:
	\centering
	\begin{longtable}{|m{\smallcellwidth}|p{\normalcellwidth}|p{\bigcellwidth}|p{\bigcellwidth}|p{\normalcellwidth}|p{\normalcellwidth}|p{\normalcellwidth}|}
	%\begin{tabularx}{\textwidth}{|r|r|r|r|r|r|r|}
		\hline
		\textbf{Mes} &
		\textbf{Contenido/Unidad Temática} &
		\textbf{Capacidades} &
		\textbf{Indicadores} &
		\textbf{Recursos Didácticos/Uso de TIC's} &
		\textbf{Instrumentos de Evaluación} &
		\textbf{Proyectos Disciplinarios} \\
		\hline
		\endhead
		% Plan de Febrero:
		Febrero &
		Introducción a la Lógica &
		\begin{itemize}
			\item Hallar soluciones creativas a situaciones problemáticas mediante el uso del razonamiento lógico.
			\item Mostrar interés o curiosidad por las cosas de su entorno.
		\end{itemize} &
		\begin{itemize}
			\item Elabora estrategias para ganar diversos tipos de juegos de mesa (Casita robada, UNO, Ajedrez, Laberintos, etc).
			\item Resuelve diversos tipos de acertijos y puzzles correctamente.
			\item Emplea diversos medios para la obtención de información (Preguntas al profesor, los compañeros o los padres, uso de lógica deductiva, experimentación, etc).
 		\end{itemize} &
		Juegos diversos, Acertijos, Puzzles, Libros de cuentos, etc. &
		Questionarios Orales, Juegos o resolución de Acertijos y Trivias. &
		 - \\
		\hline
		% Plan de Marzo:
		Marzo &
		 &
		\begin{itemize}
			\item 
		\end{itemize} &
		\begin{itemize}
			\item 
		\end{itemize} &
		  &
		  &
		 - \\
		\hline
		% Plan de Abril:
		Abril &
		 &
		\begin{itemize}
			\item 
		\end{itemize} &
		\begin{itemize}
			\item 
		\end{itemize} &
		  &
		  &
		 - \\
		\hline
		% Plan de Mayo:
		Mayo &
		 &
		\begin{itemize}
			\item 
		\end{itemize} &
		\begin{itemize}
			\item 
		\end{itemize} &
		  &
		  &
		 - \\
		\hline
		% Plan de Junio:
		Junio &
		 &
		\begin{itemize}
			\item 
		\end{itemize} &
		\begin{itemize}
			\item 
		\end{itemize} &
		  &
		  &
		 - \\
		\hline
		% Plan de Julio:
		Julio &
		 &
		\begin{itemize}
			\item 
		\end{itemize} &
		\begin{itemize}
			\item 
		\end{itemize} &
		  &
		  &
		 - \\
		\hline
		% Plan de Agosto:
		Agosto &
		 &
		\begin{itemize}
			\item 
		\end{itemize} &
		\begin{itemize}
			\item 
		\end{itemize} &
		  &
		  &
		 - \\
		\hline
		% Plan de Septiembre:
		Septiembre &
		 &
		\begin{itemize}
			\item 
		\end{itemize} &
		\begin{itemize}
			\item 
		\end{itemize} &
		  &
		  &
		 - \\
		\hline
		% Plan de Octubre:
		Octubre &
		 &
		\begin{itemize}
			\item 
		\end{itemize} &
		\begin{itemize}
			\item 
		\end{itemize} &
		  &
		  &
		 - \\
		\hline
		% Plan de Noviembre:
		Noviembre &
		 &
		\begin{itemize}
			\item 
		\end{itemize} &
		\begin{itemize}
			\item 
		\end{itemize} &
		  &
		  &
		 - \\
		\hline

	\end{longtable}
	% ----------------------------------------------------------------------------------------------------------------------------------------
\end{document}
