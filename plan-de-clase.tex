\documentclass[landscape, a4paper, 10pt]{article}
\usepackage[spanish]{babel}
\usepackage[utf8]{inputenc}
\usepackage{array}
\usepackage{graphicx}
\usepackage{tabularx}
\usepackage{longtable}
\usepackage[a4paper,margin=1.5cm,landscape]{geometry}
\usepackage{marginnote}
\usepackage{rotating}
\usepackage{blindtext}
\usepackage[dvipsnames]{xcolor}
\usepackage{tcolorbox}
\newtcolorbox{bgshadow}{
	arc=0pt,
	boxrule=2pt,
	colback=Apricot,
	width=\textheight,
	halign=center
}
% Tamaño de las celdas:
\newcommand{\smallcellwidth}{0.7in}
\newcommand{\normalcellwidth}{1.2in}
\newcommand{\bigcellwidth}{2.0in}
% Identificación:
\newcommand{\profesor}{Santiago Wu}
\newcommand{\discipline}{Robótica y Programación}
\newcommand{\currentyear}{2023}
\newcommand{\institution}{Centro Educativo y Técnico San Agustín}
\newcommand{\CEGSA}{cegsa-logo.png}
\newcommand{\CETSA}{cetsa-logo.png}
% -
\begin{document}
	% Etiqueta:
	%\reversemarginpar\marginnote{
	%	\begin{turn}{90}
	%		\begin{bgshadow}
	%			Centro Educativo y Técnico San Agustín 2023
	%		\end{bgshadow}
	%	\end{turn}
	%}
	% Cabecera:
	\begin{tabularx}{\textwidth}{ >{\raggedright\arraybackslash}X >{\centering\arraybackslash}X >{\raggedleft\arraybackslash}X }
		\includegraphics[width=0.3\textwidth]{\CEGSA} &
		\textbf{PLAN ANUAL DE CLASES} &
		\includegraphics[width=0.3\textwidth]{\CETSA}
	\end{tabularx}
	% Identificación:
	\begin{tabularx}{\textwidth}{ >{\raggedright\arraybackslash}X >{\raggedright\arraybackslash}X >{\raggedright\arraybackslash}X }
		Docente: \profesor &
		Turno: - &
		Año: \currentyear \\
		Disciplina: \discipline &
		Grado/Curso: 1er Grado &
		 \\
	\end{tabularx}
	% Contenido:
	\centering
	\begin{longtable}{|m{\smallcellwidth}|p{\normalcellwidth}|p{\bigcellwidth}|p{\bigcellwidth}|p{\normalcellwidth}|p{\normalcellwidth}|p{\normalcellwidth}|}
	%\begin{tabularx}{\textwidth}{|r|r|r|r|r|r|r|}
		\hline
		\textbf{Mes} &
		\textbf{Contenido/Unidad Temática} &
		\textbf{Capacidades} &
		\textbf{Indicadores} &
		\textbf{Recursos Didácticos/Uso de TIC's} &
		\textbf{Instrumentos de Evaluación} &
		\textbf{Proyectos Disciplinarios} \\
		\hline
		\endhead
		% Plan de Febrero:
		Febrero &
		Introducción a la Lógica &
		\begin{itemize}
			\item Hallar soluciones creativas a situaciones problemáticas mediante el uso del razonamiento lógico.
			\item Mostrar interés o curiosidad por las cosas de su entorno.
		\end{itemize} &
		\begin{itemize}
			\item Elabora estrategias para ganar diversos tipos de juegos de mesa (Casita robada, UNO, Ajedrez, Laberintos, etc).
			\item Resuelve diversos tipos de acertijos y puzzles correctamente.
			\item Emplea diversos medios para la obtención de información (Preguntas al profesor, los compañeros o los padres, uso de lógica deductiva, experimentación, etc).
 		\end{itemize} &
		Juegos diversos, Acertijos, Puzzles, Libros de cuentos, etc. &
		Questionarios Orales, Juegos o resolución de Acertijos y Trivias. &
		 - \\
		\hline
		% Plan de Marzo:
		Marzo &
		Introducción a la Lógica &
		\begin{itemize}
			\item Hallar soluciones creativas a situaciones problemáticas mediante el uso del razonamiento lógico.
			\item Mostrar interés o curiosidad por las cosas de su entorno.
		\end{itemize} &
		\begin{itemize}
			\item Elabora estrategias para ganar diversos tipos de juegos de mesa (Casita robada, UNO, Ajedrez, Laberintos, etc).
			\item Resuelve diversos tipos de acertijos y puzzles correctamente.
			\item Demuestra interés por los experimentos realizados en clase.
			\item Emplea diversos medios para la obtención de información (Preguntas al profesor, los compañeros o los padres, uso de lógica deductiva, experimentación, etc).
 		\end{itemize} &
		Juegos diversos, Acertijos, Puzzles, Libros de cuentos, Experimentos sencillos, etc. &
		Questionarios Orales y/o escritos, Juegos o resolución de Acertijos y Trivias. &
		"Llegó la caballería"\par
		Este proyecto parte del hecho de que los chicos de 1er Grado aún no poseen habilides básicas, como leer y escribir, aritmética básica, entre
		otras. Por este motivo mi propuesta en el marco del área de \discipline es ayudarlos a adquirir habilidades básicas, como el 
		pensamiento crítico, el análisis y resolución de situaciones problemáticas, y el razonamiento lógico, ayudándolos en sus 
		necesidades con otras materias mediante el empleo de juegos de mesa, acertijos y experimentos sencillos, de modo que puedan 
		sentirse motivados constantemente a investigar y aprender más sobre el mundo que los rodea.\par
		Así pues, considero que mi objetivo dentro del área de \discipline está cumplido. \\
		\hline
		% Plan de Abril:
		Abril &
		Introducción a la Lógica &
		\begin{itemize}
			\item Hallar soluciones creativas a situaciones problemáticas mediante el uso del razonamiento lógico.
			\item Mostrar interés o curiosidad por las cosas de su entorno.
		\end{itemize} &
		\begin{itemize}
			\item Elabora estrategias para ganar diversos tipos de juegos de mesa (Casita robada, UNO, Ajedrez, Laberintos, etc).
			\item Resuelve diversos tipos de acertijos y puzzles correctamente.
			\item Demuestra interés por los experimentos realizados en clase.
			\item Emplea diversos medios para la obtención de información (Preguntas al profesor, los compañeros o los padres, uso de lógica deductiva, experimentación, etc).
 		\end{itemize} &
		Juegos diversos, Acertijos, Puzzles, Libros de cuentos, Experimentos sencillos, etc. &
		Questionarios Orales, Juegos o resolución de Acertijos y Trivias. &
		"Llegó la caballería"\par
		Este proyecto parte del hecho de que los chicos de 1er Grado aún no poseen habilides básicas, como leer y escribir, aritmética básica, entre
		otras. Por este motivo mi propuesta en el marco del área de \discipline es ayudarlos a adquirir habilidades básicas, como el 
		pensamiento crítico, el análisis y resolución de situaciones problemáticas, y el razonamiento lógico, ayudándolos en sus 
		necesidades con otras materias mediante el empleo de juegos de mesa, acertijos y experimentos sencillos, de modo que puedan 
		sentirse motivados constantemente a investigar y aprender más sobre el mundo que los rodea.\par
		Así pues, considero que mi objetivo dentro del área de \discipline está cumplido. \\
		\hline
		% Plan de Mayo:
		Mayo &
		Introducción a la Lógica &
		\begin{itemize}
			\item Hallar soluciones creativas a situaciones problemáticas mediante el uso del razonamiento lógico.
			\item Mostrar interés o curiosidad por las cosas de su entorno.
		\end{itemize} &
		\begin{itemize}
			\item Elabora estrategias para ganar diversos tipos de juegos de mesa (Casita robada, UNO, Ajedrez, Laberintos, etc).
			\item Resuelve diversos tipos de acertijos y puzzles correctamente.
			\item Demuestra interés por los experimentos realizados en clase.
			\item Emplea diversos medios para la obtención de información (Preguntas al profesor, los compañeros o los padres, uso de lógica deductiva, experimentación, etc).
 		\end{itemize} &
		Juegos diversos, Acertijos, Puzzles, Libros de cuentos, Experimentos sencillos, etc. &
		Questionarios Orales, Juegos o resolución de Acertijos y Trivias. &
		"Llegó la caballería"\par
		Este proyecto parte del hecho de que los chicos de 1er Grado aún no poseen habilides básicas, como leer y escribir, aritmética básica, entre
		otras. Por este motivo mi propuesta en el marco del área de \discipline es ayudarlos a adquirir habilidades básicas, como el 
		pensamiento crítico, el análisis y resolución de situaciones problemáticas, y el razonamiento lógico, ayudándolos en sus 
		necesidades con otras materias mediante el empleo de juegos de mesa, acertijos y experimentos sencillos, de modo que puedan 
		sentirse motivados constantemente a investigar y aprender más sobre el mundo que los rodea.\par
		Así pues, considero que mi objetivo dentro del área de \discipline está cumplido. \\
		\hline
		% Plan de Junio:
		Junio &
		Introducción a la Lógica &
		\begin{itemize}
			\item Hallar soluciones creativas a situaciones problemáticas mediante el uso del razonamiento lógico.
			\item Mostrar interés o curiosidad por las cosas de su entorno.
		\end{itemize} &
		\begin{itemize}
			\item Elabora estrategias para ganar diversos tipos de juegos de mesa (Casita robada, UNO, Ajedrez, Laberintos, etc).
			\item Resuelve diversos tipos de acertijos y puzzles correctamente.
			\item Demuestra interés por los experimentos realizados en clase.
			\item Emplea diversos medios para la obtención de información (Preguntas al profesor, los compañeros o los padres, uso de lógica deductiva, experimentación, etc).
 		\end{itemize} &
		Juegos diversos, Acertijos, Puzzles, Libros de cuentos, Experimentos sencillos, etc. &
		Questionarios Orales, Juegos o resolución de Acertijos y Trivias. &
		"Llegó la caballería"\par
		Este proyecto parte del hecho de que los chicos de 1er Grado aún no poseen habilides básicas, como leer y escribir, aritmética básica, entre
		otras. Por este motivo mi propuesta en el marco del área de \discipline es ayudarlos a adquirir habilidades básicas, como el 
		pensamiento crítico, el análisis y resolución de situaciones problemáticas, y el razonamiento lógico, ayudándolos en sus 
		necesidades con otras materias mediante el empleo de juegos de mesa, acertijos y experimentos sencillos, de modo que puedan 
		sentirse motivados constantemente a investigar y aprender más sobre el mundo que los rodea.\par
		Así pues, considero que mi objetivo dentro del área de \discipline está cumplido. \\
		\hline
		% Plan de Julio:
		Julio &
		Introducción a la Lógica &
		\begin{itemize}
			\item Hallar soluciones creativas a situaciones problemáticas mediante el uso del razonamiento lógico.
			\item Mostrar interés o curiosidad por las cosas de su entorno.
		\end{itemize} &
		\begin{itemize}
			\item Elabora estrategias para ganar diversos tipos de juegos de mesa (Casita robada, UNO, Ajedrez, Laberintos, etc).
			\item Resuelve diversos tipos de acertijos y puzzles correctamente.
			\item Demuestra interés por los experimentos realizados en clase.
			\item Emplea diversos medios para la obtención de información (Preguntas al profesor, los compañeros o los padres, uso de lógica deductiva, experimentación, etc).
 		\end{itemize} &
		Juegos diversos, Acertijos, Puzzles, Libros de cuentos, Experimentos sencillos, etc. &
		Questionarios Orales, Juegos o resolución de Acertijos y Trivias. &
		"Llegó la caballería"\par
		Este proyecto parte del hecho de que los chicos de 1er Grado aún no poseen habilides básicas, como leer y escribir, aritmética básica, entre
		otras. Por este motivo mi propuesta en el marco del área de \discipline es ayudarlos a adquirir habilidades básicas, como el 
		pensamiento crítico, el análisis y resolución de situaciones problemáticas, y el razonamiento lógico, ayudándolos en sus 
		necesidades con otras materias mediante el empleo de juegos de mesa, acertijos y experimentos sencillos, de modo que puedan 
		sentirse motivados constantemente a investigar y aprender más sobre el mundo que los rodea.\par
		Así pues, considero que mi objetivo dentro del área de \discipline está cumplido. \\
		\hline
		% Plan de Agosto:
		Agosto &
		Introducción a la Lógica &
		\begin{itemize}
			\item Hallar soluciones creativas a situaciones problemáticas mediante el uso del razonamiento lógico.
			\item Mostrar interés o curiosidad por las cosas de su entorno.
		\end{itemize} &
		\begin{itemize}
			\item Elabora estrategias para ganar diversos tipos de juegos de mesa (Casita robada, UNO, Ajedrez, Laberintos, etc).
			\item Resuelve diversos tipos de acertijos y puzzles correctamente.
			\item Demuestra interés por los experimentos realizados en clase.
			\item Emplea diversos medios para la obtención de información (Preguntas al profesor, los compañeros o los padres, uso de lógica deductiva, experimentación, etc).
 		\end{itemize} &
		Juegos diversos, Acertijos, Puzzles, Libros de cuentos, Experimentos sencillos, etc. &
		Questionarios Orales, Juegos o resolución de Acertijos y Trivias. &
		"Llegó la caballería"\par
		Este proyecto parte del hecho de que los chicos de 1er Grado aún no poseen habilides básicas, como leer y escribir, aritmética básica, entre
		otras. Por este motivo mi propuesta en el marco del área de \discipline es ayudarlos a adquirir habilidades básicas, como el 
		pensamiento crítico, el análisis y resolución de situaciones problemáticas, y el razonamiento lógico, ayudándolos en sus 
		necesidades con otras materias mediante el empleo de juegos de mesa, acertijos y experimentos sencillos, de modo que puedan 
		sentirse motivados constantemente a investigar y aprender más sobre el mundo que los rodea.\par
		Así pues, considero que mi objetivo dentro del área de \discipline está cumplido. \\
		\hline
		% Plan de Septiembre:
		Septiembre &
		Introducción a la Lógica &
		\begin{itemize}
			\item Hallar soluciones creativas a situaciones problemáticas mediante el uso del razonamiento lógico.
			\item Mostrar interés o curiosidad por las cosas de su entorno.
		\end{itemize} &
		\begin{itemize}
			\item Elabora estrategias para ganar diversos tipos de juegos de mesa (Casita robada, UNO, Ajedrez, Laberintos, etc).
			\item Resuelve diversos tipos de acertijos y puzzles correctamente.
			\item Demuestra interés por los experimentos realizados en clase.
			\item Emplea diversos medios para la obtención de información (Preguntas al profesor, los compañeros o los padres, uso de lógica deductiva, experimentación, etc).
 		\end{itemize} &
		Juegos diversos, Acertijos, Puzzles, Libros de cuentos, Experimentos sencillos, etc. &
		Questionarios Orales, Juegos o resolución de Acertijos y Trivias. &
		"Llegó la caballería"\par
		Este proyecto parte del hecho de que los chicos de 1er Grado aún no poseen habilides básicas, como leer y escribir, aritmética básica, entre
		otras. Por este motivo mi propuesta en el marco del área de \discipline es ayudarlos a adquirir habilidades básicas, como el 
		pensamiento crítico, el análisis y resolución de situaciones problemáticas, y el razonamiento lógico, ayudándolos en sus 
		necesidades con otras materias mediante el empleo de juegos de mesa, acertijos y experimentos sencillos, de modo que puedan 
		sentirse motivados constantemente a investigar y aprender más sobre el mundo que los rodea.\par
		Así pues, considero que mi objetivo dentro del área de \discipline está cumplido. \\
		\hline
		% Plan de Octubre:
		Octubre &
		Introducción a la Lógica &
		\begin{itemize}
			\item Hallar soluciones creativas a situaciones problemáticas mediante el uso del razonamiento lógico.
			\item Mostrar interés o curiosidad por las cosas de su entorno.
		\end{itemize} &
		\begin{itemize}
			\item Elabora estrategias para ganar diversos tipos de juegos de mesa (Casita robada, UNO, Ajedrez, Laberintos, etc).
			\item Resuelve diversos tipos de acertijos y puzzles correctamente.
			\item Demuestra interés por los experimentos realizados en clase.
			\item Emplea diversos medios para la obtención de información (Preguntas al profesor, los compañeros o los padres, uso de lógica deductiva, experimentación, etc).
 		\end{itemize} &
		Juegos diversos, Acertijos, Puzzles, Libros de cuentos, Experimentos sencillos, etc. &
		Questionarios Orales, Juegos o resolución de Acertijos y Trivias. &
		"Llegó la caballería"\par
		Este proyecto parte del hecho de que los chicos de 1er Grado aún no poseen habilides básicas, como leer y escribir, aritmética básica, entre
		otras. Por este motivo mi propuesta en el marco del área de \discipline es ayudarlos a adquirir habilidades básicas, como el 
		pensamiento crítico, el análisis y resolución de situaciones problemáticas, y el razonamiento lógico, ayudándolos en sus 
		necesidades con otras materias mediante el empleo de juegos de mesa, acertijos y experimentos sencillos, de modo que puedan 
		sentirse motivados constantemente a investigar y aprender más sobre el mundo que los rodea.\par
		Así pues, considero que mi objetivo dentro del área de \discipline está cumplido. \\
		\hline
		% Plan de Noviembre:
		Noviembre &
		Introducción a la Lógica &
		\begin{itemize}
			\item Hallar soluciones creativas a situaciones problemáticas mediante el uso del razonamiento lógico.
			\item Mostrar interés o curiosidad por las cosas de su entorno.
		\end{itemize} &
		\begin{itemize}
			\item Elabora estrategias para ganar diversos tipos de juegos de mesa (Casita robada, UNO, Ajedrez, Laberintos, etc).
			\item Resuelve diversos tipos de acertijos y puzzles correctamente.
			\item Demuestra interés por los experimentos realizados en clase.
			\item Emplea diversos medios para la obtención de información (Preguntas al profesor, los compañeros o los padres, uso de lógica deductiva, experimentación, etc).
 		\end{itemize} &
		Juegos diversos, Acertijos, Puzzles, Libros de cuentos, Experimentos sencillos, etc. &
		Questionarios Orales, Juegos o resolución de Acertijos y Trivias. &
		"Llegó la caballería"\par
		Este proyecto parte del hecho de que los chicos de 1er Grado aún no poseen habilides básicas, como leer y escribir, aritmética básica, entre
		otras. Por este motivo mi propuesta en el marco del área de \discipline es ayudarlos a adquirir habilidades básicas, como el 
		pensamiento crítico, el análisis y resolución de situaciones problemáticas, y el razonamiento lógico, ayudándolos en sus 
		necesidades con otras materias mediante el empleo de juegos de mesa, acertijos y experimentos sencillos, de modo que puedan 
		sentirse motivados constantemente a investigar y aprender más sobre el mundo que los rodea.\par
		Así pues, considero que mi objetivo dentro del área de \discipline está cumplido. \\
		\hline

	\end{longtable}
	\pagebreak[4]
	% ----------------------------------------------------------------------------------------------------------------------------------------
	% Cabecera:
	\begin{tabularx}{\textwidth}{ >{\raggedright\arraybackslash}X >{\centering\arraybackslash}X >{\raggedleft\arraybackslash}X }
		\includegraphics[width=0.3\textwidth]{\CEGSA} &
		\textbf{PLAN ANUAL DE CLASES} &
		\includegraphics[width=0.3\textwidth]{\CETSA}
	\end{tabularx}
	% Identificación:
	\begin{tabularx}{\textwidth}{ >{\raggedright\arraybackslash}X >{\raggedright\arraybackslash}X >{\raggedright\arraybackslash}X }
		Docente: \profesor &
		Turno: - &
		Año: \currentyear \\
		Disciplina: \discipline &
		Grado/Curso: 2do Grado &
		 \\
	\end{tabularx}
	% Contenido:
	\centering
	\begin{longtable}{|m{\smallcellwidth}|p{\normalcellwidth}|p{\bigcellwidth}|p{\bigcellwidth}|p{\normalcellwidth}|p{\normalcellwidth}|p{\normalcellwidth}|}
	%\begin{tabularx}{\textwidth}{|r|r|r|r|r|r|r|}
		\hline
		\textbf{Mes} &
		\textbf{Contenido/Unidad Temática} &
		\textbf{Capacidades} &
		\textbf{Indicadores} &
		\textbf{Recursos Didácticos/Uso de TIC's} &
		\textbf{Instrumentos de Evaluación} &
		\textbf{Proyectos Disciplinarios} \\
		\hline
		\endhead
		% Plan de Febrero:
		Febrero &
		 &
		\begin{itemize}
			\item 
		\end{itemize} &
		\begin{itemize}
			\item 
		\end{itemize} &
		  &
		  &
		 - \\
		\hline
		% Plan de Marzo:
		Marzo &
		 &
		\begin{itemize}
			\item 
		\end{itemize} &
		\begin{itemize}
			\item 
		\end{itemize} &
		  &
		  &
		 - \\
		\hline
		% Plan de Abril:
		Abril &
		 &
		\begin{itemize}
			\item 
		\end{itemize} &
		\begin{itemize}
			\item 
		\end{itemize} &
		  &
		  &
		 - \\
		\hline
		% Plan de Mayo:
		Mayo &
		 &
		\begin{itemize}
			\item 
		\end{itemize} &
		\begin{itemize}
			\item 
		\end{itemize} &
		  &
		  &
		 - \\
		\hline
		% Plan de Junio:
		Junio &
		 &
		\begin{itemize}
			\item 
		\end{itemize} &
		\begin{itemize}
			\item 
		\end{itemize} &
		  &
		  &
		 - \\
		\hline
		% Plan de Julio:
		Julio &
		 &
		\begin{itemize}
			\item 
		\end{itemize} &
		\begin{itemize}
			\item 
		\end{itemize} &
		  &
		  &
		 - \\
		\hline
		% Plan de Agosto:
		Agosto &
		 &
		\begin{itemize}
			\item 
		\end{itemize} &
		\begin{itemize}
			\item 
		\end{itemize} &
		  &
		  &
		 - \\
		\hline
		% Plan de Septiembre:
		Septiembre &
		 &
		\begin{itemize}
			\item 
		\end{itemize} &
		\begin{itemize}
			\item 
		\end{itemize} &
		  &
		  &
		 - \\
		\hline
		% Plan de Octubre:
		Octubre &
		 &
		\begin{itemize}
			\item 
		\end{itemize} &
		\begin{itemize}
			\item 
		\end{itemize} &
		  &
		  &
		 - \\
		\hline
		% Plan de Noviembre:
		Noviembre &
		 &
		\begin{itemize}
			\item 
		\end{itemize} &
		\begin{itemize}
			\item 
		\end{itemize} &
		  &
		  &
		 - \\
		\hline

	\end{longtable}
	\pagebreak[4]
	% ----------------------------------------------------------------------------------------------------------------------------------------
	% Cabecera:
	\begin{tabularx}{\textwidth}{ >{\raggedright\arraybackslash}X >{\centering\arraybackslash}X >{\raggedleft\arraybackslash}X }
		\includegraphics[width=0.3\textwidth]{\CEGSA} &
		\textbf{PLAN ANUAL DE CLASES} &
		\includegraphics[width=0.3\textwidth]{\CETSA}
	\end{tabularx}
	% Identificación:
	\begin{tabularx}{\textwidth}{ >{\raggedright\arraybackslash}X >{\raggedright\arraybackslash}X >{\raggedright\arraybackslash}X }
		Docente: \profesor &
		Turno: - &
		Año: \currentyear \\
		Disciplina: \discipline &
		Grado/Curso: 3er Grado &
		 \\
	\end{tabularx}
	% Contenido:
	\centering
	\begin{longtable}{|m{\smallcellwidth}|p{\normalcellwidth}|p{\bigcellwidth}|p{\bigcellwidth}|p{\normalcellwidth}|p{\normalcellwidth}|p{\normalcellwidth}|}
	%\begin{tabularx}{\textwidth}{|r|r|r|r|r|r|r|}
		\hline
		\textbf{Mes} &
		\textbf{Contenido/Unidad Temática} &
		\textbf{Capacidades} &
		\textbf{Indicadores} &
		\textbf{Recursos Didácticos/Uso de TIC's} &
		\textbf{Instrumentos de Evaluación} &
		\textbf{Proyectos Disciplinarios} \\
		\hline
		\endhead
		% Plan de Febrero:
		Febrero &
		 &
		\begin{itemize}
			\item 
		\end{itemize} &
		\begin{itemize}
			\item 
		\end{itemize} &
		  &
		  &
		 - \\
		\hline
		% Plan de Marzo:
		Marzo &
		 &
		\begin{itemize}
			\item 
		\end{itemize} &
		\begin{itemize}
			\item 
		\end{itemize} &
		  &
		  &
		 - \\
		\hline
		% Plan de Abril:
		Abril &
		 &
		\begin{itemize}
			\item 
		\end{itemize} &
		\begin{itemize}
			\item 
		\end{itemize} &
		  &
		  &
		 - \\
		\hline
		% Plan de Mayo:
		Mayo &
		 &
		\begin{itemize}
			\item 
		\end{itemize} &
		\begin{itemize}
			\item 
		\end{itemize} &
		  &
		  &
		 - \\
		\hline
		% Plan de Junio:
		Junio &
		 &
		\begin{itemize}
			\item 
		\end{itemize} &
		\begin{itemize}
			\item 
		\end{itemize} &
		  &
		  &
		 - \\
		\hline
		% Plan de Julio:
		Julio &
		 &
		\begin{itemize}
			\item 
		\end{itemize} &
		\begin{itemize}
			\item 
		\end{itemize} &
		  &
		  &
		 - \\
		\hline
		% Plan de Agosto:
		Agosto &
		 &
		\begin{itemize}
			\item 
		\end{itemize} &
		\begin{itemize}
			\item 
		\end{itemize} &
		  &
		  &
		 - \\
		\hline
		% Plan de Septiembre:
		Septiembre &
		 &
		\begin{itemize}
			\item 
		\end{itemize} &
		\begin{itemize}
			\item 
		\end{itemize} &
		  &
		  &
		 - \\
		\hline
		% Plan de Octubre:
		Octubre &
		 &
		\begin{itemize}
			\item 
		\end{itemize} &
		\begin{itemize}
			\item 
		\end{itemize} &
		  &
		  &
		 - \\
		\hline
		% Plan de Noviembre:
		Noviembre &
		 &
		\begin{itemize}
			\item 
		\end{itemize} &
		\begin{itemize}
			\item 
		\end{itemize} &
		  &
		  &
		 - \\
		\hline

	\end{longtable}
	\pagebreak[4]
	% ----------------------------------------------------------------------------------------------------------------------------------------
	% Cabecera:
	\begin{tabularx}{\textwidth}{ >{\raggedright\arraybackslash}X >{\centering\arraybackslash}X >{\raggedleft\arraybackslash}X }
		\includegraphics[width=0.3\textwidth]{\CEGSA} &
		\textbf{PLAN ANUAL DE CLASES} &
		\includegraphics[width=0.3\textwidth]{\CETSA}
	\end{tabularx}
	% Identificación:
	\begin{tabularx}{\textwidth}{ >{\raggedright\arraybackslash}X >{\raggedright\arraybackslash}X >{\raggedright\arraybackslash}X }
		Docente: \profesor &
		Turno: - &
		Año: \currentyear \\
		Disciplina: \discipline &
		Grado/Curso: 4to Grado &
		 \\
	\end{tabularx}
	% Contenido:
	\centering
	\begin{longtable}{|m{\smallcellwidth}|p{\normalcellwidth}|p{\bigcellwidth}|p{\bigcellwidth}|p{\normalcellwidth}|p{\normalcellwidth}|p{\normalcellwidth}|}
	%\begin{tabularx}{\textwidth}{|r|r|r|r|r|r|r|}
		\hline
		\textbf{Mes} &
		\textbf{Contenido/Unidad Temática} &
		\textbf{Capacidades} &
		\textbf{Indicadores} &
		\textbf{Recursos Didácticos/Uso de TIC's} &
		\textbf{Instrumentos de Evaluación} &
		\textbf{Proyectos Disciplinarios} \\
		\hline
		\endhead
		% Plan de Febrero:
		Febrero &
		 &
		\begin{itemize}
			\item 
		\end{itemize} &
		\begin{itemize}
			\item 
		\end{itemize} &
		  &
		  &
		 - \\
		\hline
		% Plan de Marzo:
		Marzo &
		 &
		\begin{itemize}
			\item 
		\end{itemize} &
		\begin{itemize}
			\item 
		\end{itemize} &
		  &
		  &
		 - \\
		\hline
		% Plan de Abril:
		Abril &
		 &
		\begin{itemize}
			\item 
		\end{itemize} &
		\begin{itemize}
			\item 
		\end{itemize} &
		  &
		  &
		 - \\
		\hline
		% Plan de Mayo:
		Mayo &
		 &
		\begin{itemize}
			\item 
		\end{itemize} &
		\begin{itemize}
			\item 
		\end{itemize} &
		  &
		  &
		 - \\
		\hline
		% Plan de Junio:
		Junio &
		 &
		\begin{itemize}
			\item 
		\end{itemize} &
		\begin{itemize}
			\item 
		\end{itemize} &
		  &
		  &
		 - \\
		\hline
		% Plan de Julio:
		Julio &
		 &
		\begin{itemize}
			\item 
		\end{itemize} &
		\begin{itemize}
			\item 
		\end{itemize} &
		  &
		  &
		 - \\
		\hline
		% Plan de Agosto:
		Agosto &
		 &
		\begin{itemize}
			\item 
		\end{itemize} &
		\begin{itemize}
			\item 
		\end{itemize} &
		  &
		  &
		 - \\
		\hline
		% Plan de Septiembre:
		Septiembre &
		 &
		\begin{itemize}
			\item 
		\end{itemize} &
		\begin{itemize}
			\item 
		\end{itemize} &
		  &
		  &
		 - \\
		\hline
		% Plan de Octubre:
		Octubre &
		 &
		\begin{itemize}
			\item 
		\end{itemize} &
		\begin{itemize}
			\item 
		\end{itemize} &
		  &
		  &
		 - \\
		\hline
		% Plan de Noviembre:
		Noviembre &
		 &
		\begin{itemize}
			\item 
		\end{itemize} &
		\begin{itemize}
			\item 
		\end{itemize} &
		  &
		  &
		 - \\
		\hline

	\end{longtable}
	\pagebreak[4]
	% ----------------------------------------------------------------------------------------------------------------------------------------
	% Cabecera:
	\begin{tabularx}{\textwidth}{ >{\raggedright\arraybackslash}X >{\centering\arraybackslash}X >{\raggedleft\arraybackslash}X }
		\includegraphics[width=0.3\textwidth]{\CEGSA} &
		\textbf{PLAN ANUAL DE CLASES} &
		\includegraphics[width=0.3\textwidth]{\CETSA}
	\end{tabularx}
	% Identificación:
	\begin{tabularx}{\textwidth}{ >{\raggedright\arraybackslash}X >{\raggedright\arraybackslash}X >{\raggedright\arraybackslash}X }
		Docente: \profesor &
		Turno: - &
		Año: \currentyear \\
		Disciplina: \discipline &
		Grado/Curso: 5to Grado &
		 \\
	\end{tabularx}
	% Contenido:
	\centering
	\begin{longtable}{|m{\smallcellwidth}|p{\normalcellwidth}|p{\bigcellwidth}|p{\bigcellwidth}|p{\normalcellwidth}|p{\normalcellwidth}|p{\normalcellwidth}|}
	%\begin{tabularx}{\textwidth}{|r|r|r|r|r|r|r|}
		\hline
		\textbf{Mes} &
		\textbf{Contenido/Unidad Temática} &
		\textbf{Capacidades} &
		\textbf{Indicadores} &
		\textbf{Recursos Didácticos/Uso de TIC's} &
		\textbf{Instrumentos de Evaluación} &
		\textbf{Proyectos Disciplinarios} \\
		\hline
		\endhead
		% Plan de Febrero:
		Febrero &
		 &
		\begin{itemize}
			\item 
		\end{itemize} &
		\begin{itemize}
			\item 
		\end{itemize} &
		  &
		  &
		 - \\
		\hline
		% Plan de Marzo:
		Marzo &
		 &
		\begin{itemize}
			\item 
		\end{itemize} &
		\begin{itemize}
			\item 
		\end{itemize} &
		  &
		  &
		 - \\
		\hline
		% Plan de Abril:
		Abril &
		 &
		\begin{itemize}
			\item 
		\end{itemize} &
		\begin{itemize}
			\item 
		\end{itemize} &
		  &
		  &
		 - \\
		\hline
		% Plan de Mayo:
		Mayo &
		 &
		\begin{itemize}
			\item 
		\end{itemize} &
		\begin{itemize}
			\item 
		\end{itemize} &
		  &
		  &
		 - \\
		\hline
		% Plan de Junio:
		Junio &
		 &
		\begin{itemize}
			\item 
		\end{itemize} &
		\begin{itemize}
			\item 
		\end{itemize} &
		  &
		  &
		 - \\
		\hline
		% Plan de Julio:
		Julio &
		 &
		\begin{itemize}
			\item 
		\end{itemize} &
		\begin{itemize}
			\item 
		\end{itemize} &
		  &
		  &
		 - \\
		\hline
		% Plan de Agosto:
		Agosto &
		 &
		\begin{itemize}
			\item 
		\end{itemize} &
		\begin{itemize}
			\item 
		\end{itemize} &
		  &
		  &
		 - \\
		\hline
		% Plan de Septiembre:
		Septiembre &
		 &
		\begin{itemize}
			\item 
		\end{itemize} &
		\begin{itemize}
			\item 
		\end{itemize} &
		  &
		  &
		 - \\
		\hline
		% Plan de Octubre:
		Octubre &
		 &
		\begin{itemize}
			\item 
		\end{itemize} &
		\begin{itemize}
			\item 
		\end{itemize} &
		  &
		  &
		 - \\
		\hline
		% Plan de Noviembre:
		Noviembre &
		 &
		\begin{itemize}
			\item 
		\end{itemize} &
		\begin{itemize}
			\item 
		\end{itemize} &
		  &
		  &
		 - \\
		\hline

	\end{longtable}
	\pagebreak[4]
	% ----------------------------------------------------------------------------------------------------------------------------------------
	% Cabecera:
	\begin{tabularx}{\textwidth}{ >{\raggedright\arraybackslash}X >{\centering\arraybackslash}X >{\raggedleft\arraybackslash}X }
		\includegraphics[width=0.3\textwidth]{\CEGSA} &
		\textbf{PLAN ANUAL DE CLASES} &
		\includegraphics[width=0.3\textwidth]{\CETSA}
	\end{tabularx}
	% Identificación:
	\begin{tabularx}{\textwidth}{ >{\raggedright\arraybackslash}X >{\raggedright\arraybackslash}X >{\raggedright\arraybackslash}X }
		Docente: \profesor &
		Turno: - &
		Año: \currentyear \\
		Disciplina: \discipline &
		Grado/Curso: 6to Grado &
		 \\
	\end{tabularx}
	% Contenido:
	\centering
	\begin{longtable}{|m{\smallcellwidth}|p{\normalcellwidth}|p{\bigcellwidth}|p{\bigcellwidth}|p{\normalcellwidth}|p{\normalcellwidth}|p{\normalcellwidth}|}
	%\begin{tabularx}{\textwidth}{|r|r|r|r|r|r|r|}
		\hline
		\textbf{Mes} &
		\textbf{Contenido/Unidad Temática} &
		\textbf{Capacidades} &
		\textbf{Indicadores} &
		\textbf{Recursos Didácticos/Uso de TIC's} &
		\textbf{Instrumentos de Evaluación} &
		\textbf{Proyectos Disciplinarios} \\
		\hline
		\endhead
		% Plan de Febrero:
		Febrero &
		 &
		\begin{itemize}
			\item 
		\end{itemize} &
		\begin{itemize}
			\item 
		\end{itemize} &
		  &
		  &
		 - \\
		\hline
		% Plan de Marzo:
		Marzo &
		 &
		\begin{itemize}
			\item 
		\end{itemize} &
		\begin{itemize}
			\item 
		\end{itemize} &
		  &
		  &
		 - \\
		\hline
		% Plan de Abril:
		Abril &
		 &
		\begin{itemize}
			\item 
		\end{itemize} &
		\begin{itemize}
			\item 
		\end{itemize} &
		  &
		  &
		 - \\
		\hline
		% Plan de Mayo:
		Mayo &
		 &
		\begin{itemize}
			\item 
		\end{itemize} &
		\begin{itemize}
			\item 
		\end{itemize} &
		  &
		  &
		 - \\
		\hline
		% Plan de Junio:
		Junio &
		 &
		\begin{itemize}
			\item 
		\end{itemize} &
		\begin{itemize}
			\item 
		\end{itemize} &
		  &
		  &
		 - \\
		\hline
		% Plan de Julio:
		Julio &
		 &
		\begin{itemize}
			\item 
		\end{itemize} &
		\begin{itemize}
			\item 
		\end{itemize} &
		  &
		  &
		 - \\
		\hline
		% Plan de Agosto:
		Agosto &
		 &
		\begin{itemize}
			\item 
		\end{itemize} &
		\begin{itemize}
			\item 
		\end{itemize} &
		  &
		  &
		 - \\
		\hline
		% Plan de Septiembre:
		Septiembre &
		 &
		\begin{itemize}
			\item 
		\end{itemize} &
		\begin{itemize}
			\item 
		\end{itemize} &
		  &
		  &
		 - \\
		\hline
		% Plan de Octubre:
		Octubre &
		 &
		\begin{itemize}
			\item 
		\end{itemize} &
		\begin{itemize}
			\item 
		\end{itemize} &
		  &
		  &
		 - \\
		\hline
		% Plan de Noviembre:
		Noviembre &
		 &
		\begin{itemize}
			\item 
		\end{itemize} &
		\begin{itemize}
			\item 
		\end{itemize} &
		  &
		  &
		 - \\
		\hline

	\end{longtable}
	\pagebreak[4]
	% ----------------------------------------------------------------------------------------------------------------------------------------
	% Cabecera:
	\begin{tabularx}{\textwidth}{ >{\raggedright\arraybackslash}X >{\centering\arraybackslash}X >{\raggedleft\arraybackslash}X }
		\includegraphics[width=0.3\textwidth]{\CEGSA} &
		\textbf{PLAN ANUAL DE CLASES} &
		\includegraphics[width=0.3\textwidth]{\CETSA}
	\end{tabularx}
	% Identificación:
	\begin{tabularx}{\textwidth}{ >{\raggedright\arraybackslash}X >{\raggedright\arraybackslash}X >{\raggedright\arraybackslash}X }
		Docente: \profesor &
		Turno: - &
		Año: \currentyear \\
		Disciplina: \discipline &
		Grado/Curso: 7mo Grado &
		 \\
	\end{tabularx}
	% Contenido:
	\centering
	\begin{longtable}{|m{\smallcellwidth}|p{\normalcellwidth}|p{\bigcellwidth}|p{\bigcellwidth}|p{\normalcellwidth}|p{\normalcellwidth}|p{\normalcellwidth}|}
	
	%\begin{tabularx}{\textwidth}{|r|r|r|r|r|r|r|}
		\hline
		\textbf{Mes} &
		\textbf{Contenido/Unidad Temática} &
		\textbf{Capacidades} &
		\textbf{Indicadores} &
		\textbf{Recursos Didácticos/Uso de TIC's} &
		\textbf{Instrumentos de Evaluación} &
		\textbf{Proyectos Disciplinarios} \\
		\hline
		\endhead
		% Plan de Febrero:
		Febrero &
		 &
		\begin{itemize}
			\item 
		\end{itemize} &
		\begin{itemize}
			\item 
		\end{itemize} &
		  &
		  &
		 - \\
		\hline
		% Plan de Marzo:
		Marzo &
		 &
		\begin{itemize}
			\item 
		\end{itemize} &
		\begin{itemize}
			\item 
		\end{itemize} &
		  &
		  &
		 - \\
		\hline
		% Plan de Abril:
		Abril &
		 &
		\begin{itemize}
			\item 
		\end{itemize} &
		\begin{itemize}
			\item 
		\end{itemize} &
		  &
		  &
		 - \\
		\hline
		% Plan de Mayo:
		Mayo &
		 &
		\begin{itemize}
			\item 
		\end{itemize} &
		\begin{itemize}
			\item 
		\end{itemize} &
		  &
		  &
		 - \\
		\hline
		% Plan de Junio:
		Junio &
		 &
		\begin{itemize}
			\item 
		\end{itemize} &
		\begin{itemize}
			\item 
		\end{itemize} &
		  &
		  &
		 - \\
		\hline
		% Plan de Julio:
		Julio &
		 &
		\begin{itemize}
			\item 
		\end{itemize} &
		\begin{itemize}
			\item 
		\end{itemize} &
		  &
		  &
		 - \\
		\hline
		% Plan de Agosto:
		Agosto &
		 &
		\begin{itemize}
			\item 
		\end{itemize} &
		\begin{itemize}
			\item 
		\end{itemize} &
		  &
		  &
		 - \\
		\hline
		% Plan de Septiembre:
		Septiembre &
		 &
		\begin{itemize}
			\item 
		\end{itemize} &
		\begin{itemize}
			\item 
		\end{itemize} &
		  &
		  &
		 - \\
		\hline
		% Plan de Octubre:
		Octubre &
		 &
		\begin{itemize}
			\item 
		\end{itemize} &
		\begin{itemize}
			\item 
		\end{itemize} &
		  &
		  &
		 - \\
		\hline
		% Plan de Noviembre:
		Noviembre &
		 &
		\begin{itemize}
			\item 
		\end{itemize} &
		\begin{itemize}
			\item 
		\end{itemize} &
		  &
		  &
		 - \\
		\hline

	\end{longtable}
	\pagebreak[4]
	% ----------------------------------------------------------------------------------------------------------------------------------------
	% Cabecera:
	\begin{tabularx}{\textwidth}{ >{\raggedright\arraybackslash}X >{\centering\arraybackslash}X >{\raggedleft\arraybackslash}X }
		\includegraphics[width=0.3\textwidth]{\CEGSA} &
		\textbf{PLAN ANUAL DE CLASES} &
		\includegraphics[width=0.3\textwidth]{\CETSA}
	\end{tabularx}
	% Identificación:
	\begin{tabularx}{\textwidth}{ >{\raggedright\arraybackslash}X >{\raggedright\arraybackslash}X >{\raggedright\arraybackslash}X }
		Docente: \profesor &
		Turno: - &
		Año: \currentyear \\
		Disciplina: \discipline &
		Grado/Curso: 8vo Grado &
		 \\
	\end{tabularx}
	% Contenido:
	\centering
	\begin{longtable}{|m{\smallcellwidth}|p{\normalcellwidth}|p{\bigcellwidth}|p{\bigcellwidth}|p{\normalcellwidth}|p{\normalcellwidth}|p{\normalcellwidth}|}
	%\begin{tabularx}{\textwidth}{|r|r|r|r|r|r|r|}
		\hline
		\textbf{Mes} &
		\textbf{Contenido/Unidad Temática} &
		\textbf{Capacidades} &
		\textbf{Indicadores} &
		\textbf{Recursos Didácticos/Uso de TIC's} &
		\textbf{Instrumentos de Evaluación} &
		\textbf{Proyectos Disciplinarios} \\
		\hline
		\endhead
		% Plan de Febrero:
		Febrero &
		 &
		\begin{itemize}
			\item 
		\end{itemize} &
		\begin{itemize}
			\item 
		\end{itemize} &
		  &
		  &
		 - \\
		\hline
		% Plan de Marzo:
		Marzo &
		 &
		\begin{itemize}
			\item 
		\end{itemize} &
		\begin{itemize}
			\item 
		\end{itemize} &
		  &
		  &
		 - \\
		\hline
		% Plan de Abril:
		Abril &
		 &
		\begin{itemize}
			\item 
		\end{itemize} &
		\begin{itemize}
			\item 
		\end{itemize} &
		  &
		  &
		 - \\
		\hline
		% Plan de Mayo:
		Mayo &
		 &
		\begin{itemize}
			\item 
		\end{itemize} &
		\begin{itemize}
			\item 
		\end{itemize} &
		  &
		  &
		 - \\
		\hline
		% Plan de Junio:
		Junio &
		 &
		\begin{itemize}
			\item 
		\end{itemize} &
		\begin{itemize}
			\item 
		\end{itemize} &
		  &
		  &
		 - \\
		\hline
		% Plan de Julio:
		Julio &
		 &
		\begin{itemize}
			\item 
		\end{itemize} &
		\begin{itemize}
			\item 
		\end{itemize} &
		  &
		  &
		 - \\
		\hline
		% Plan de Agosto:
		Agosto &
		 &
		\begin{itemize}
			\item 
		\end{itemize} &
		\begin{itemize}
			\item 
		\end{itemize} &
		  &
		  &
		 - \\
		\hline
		% Plan de Septiembre:
		Septiembre &
		 &
		\begin{itemize}
			\item 
		\end{itemize} &
		\begin{itemize}
			\item 
		\end{itemize} &
		  &
		  &
		 - \\
		\hline
		% Plan de Octubre:
		Octubre &
		 &
		\begin{itemize}
			\item 
		\end{itemize} &
		\begin{itemize}
			\item 
		\end{itemize} &
		  &
		  &
		 - \\
		\hline
		% Plan de Noviembre:
		Noviembre &
		 &
		\begin{itemize}
			\item 
		\end{itemize} &
		\begin{itemize}
			\item 
		\end{itemize} &
		  &
		  &
		 - \\
		\hline

	\end{longtable}
	\pagebreak[4]
	% ----------------------------------------------------------------------------------------------------------------------------------------
	% Cabecera:
	\begin{tabularx}{\textwidth}{ >{\raggedright\arraybackslash}X >{\centering\arraybackslash}X >{\raggedleft\arraybackslash}X }
		\includegraphics[width=0.3\textwidth]{\CEGSA} &
		\textbf{PLAN ANUAL DE CLASES} &
		\includegraphics[width=0.3\textwidth]{\CETSA}
	\end{tabularx}
	% Identificación:
	\begin{tabularx}{\textwidth}{ >{\raggedright\arraybackslash}X >{\raggedright\arraybackslash}X >{\raggedright\arraybackslash}X }
		Docente: \profesor &
		Turno: - &
		Año: \currentyear \\
		Disciplina: \discipline &
		Grado/Curso: 9no Grado &
		 \\
	\end{tabularx}
	% Contenido:
	\centering
	\begin{longtable}{|m{\smallcellwidth}|p{\normalcellwidth}|p{\bigcellwidth}|p{\bigcellwidth}|p{\normalcellwidth}|p{\normalcellwidth}|p{\normalcellwidth}|}
	%\begin{tabularx}{\textwidth}{|r|r|r|r|r|r|r|}
		\hline
		\textbf{Mes} &
		\textbf{Contenido/Unidad Temática} &
		\textbf{Capacidades} &
		\textbf{Indicadores} &
		\textbf{Recursos Didácticos/Uso de TIC's} &
		\textbf{Instrumentos de Evaluación} &
		\textbf{Proyectos Disciplinarios} \\
		\hline
		\endhead
		% Plan de Febrero:
		Febrero &
		 &
		\begin{itemize}
			\item 
		\end{itemize} &
		\begin{itemize}
			\item 
		\end{itemize} &
		  &
		  &
		 - \\
		\hline
		% Plan de Marzo:
		Marzo &
		 &
		\begin{itemize}
			\item 
		\end{itemize} &
		\begin{itemize}
			\item 
		\end{itemize} &
		  &
		  &
		 - \\
		\hline
		% Plan de Abril:
		Abril &
		 &
		\begin{itemize}
			\item 
		\end{itemize} &
		\begin{itemize}
			\item 
		\end{itemize} &
		  &
		  &
		 - \\
		\hline
		% Plan de Mayo:
		Mayo &
		 &
		\begin{itemize}
			\item 
		\end{itemize} &
		\begin{itemize}
			\item 
		\end{itemize} &
		  &
		  &
		 - \\
		\hline
		% Plan de Junio:
		Junio &
		 &
		\begin{itemize}
			\item 
		\end{itemize} &
		\begin{itemize}
			\item 
		\end{itemize} &
		  &
		  &
		 - \\
		\hline
		% Plan de Julio:
		Julio &
		 &
		\begin{itemize}
			\item 
		\end{itemize} &
		\begin{itemize}
			\item 
		\end{itemize} &
		  &
		  &
		 - \\
		\hline
		% Plan de Agosto:
		Agosto &
		 &
		\begin{itemize}
			\item 
		\end{itemize} &
		\begin{itemize}
			\item 
		\end{itemize} &
		  &
		  &
		 - \\
		\hline
		% Plan de Septiembre:
		Septiembre &
		 &
		\begin{itemize}
			\item 
		\end{itemize} &
		\begin{itemize}
			\item 
		\end{itemize} &
		  &
		  &
		 - \\
		\hline
		% Plan de Octubre:
		Octubre &
		 &
		\begin{itemize}
			\item 
		\end{itemize} &
		\begin{itemize}
			\item 
		\end{itemize} &
		  &
		  &
		 - \\
		\hline
		% Plan de Noviembre:
		Noviembre &
		 &
		\begin{itemize}
			\item 
		\end{itemize} &
		\begin{itemize}
			\item 
		\end{itemize} &
		  &
		  &
		 - \\
		\hline

	\end{longtable}
	\pagebreak[4]
	% ----------------------------------------------------------------------------------------------------------------------------------------
	% Cabecera:
	\begin{tabularx}{\textwidth}{ >{\raggedright\arraybackslash}X >{\centering\arraybackslash}X >{\raggedleft\arraybackslash}X }
		\includegraphics[width=0.3\textwidth]{\CEGSA} &
		\textbf{PLAN ANUAL DE CLASES} &
		\includegraphics[width=0.3\textwidth]{\CETSA}
	\end{tabularx}
	% Identificación:
	\begin{tabularx}{\textwidth}{ >{\raggedright\arraybackslash}X >{\raggedright\arraybackslash}X >{\raggedright\arraybackslash}X }
		Docente: \profesor &
		Turno: - &
		Año: \currentyear \\
		Disciplina: \discipline &
		Grado/Curso: 1er Curso &
		 \\
	\end{tabularx}
	% Contenido:
	\centering
	\begin{longtable}{|m{\smallcellwidth}|p{\normalcellwidth}|p{\bigcellwidth}|p{\bigcellwidth}|p{\normalcellwidth}|p{\normalcellwidth}|p{\normalcellwidth}|}
	%\begin{tabularx}{\textwidth}{|r|r|r|r|r|r|r|}
		\hline
		\textbf{Mes} &
		\textbf{Contenido/Unidad Temática} &
		\textbf{Capacidades} &
		\textbf{Indicadores} &
		\textbf{Recursos Didácticos/Uso de TIC's} &
		\textbf{Instrumentos de Evaluación} &
		\textbf{Proyectos Disciplinarios} \\
		\hline
		\endhead
		% Plan de Febrero:
		Febrero &
		 &
		\begin{itemize}
			\item 
		\end{itemize} &
		\begin{itemize}
			\item 
		\end{itemize} &
		  &
		  &
		 - \\
		\hline
		% Plan de Marzo:
		Marzo &
		 &
		\begin{itemize}
			\item 
		\end{itemize} &
		\begin{itemize}
			\item 
		\end{itemize} &
		  &
		  &
		 - \\
		\hline
		% Plan de Abril:
		Abril &
		 &
		\begin{itemize}
			\item 
		\end{itemize} &
		\begin{itemize}
			\item 
		\end{itemize} &
		  &
		  &
		 - \\
		\hline
		% Plan de Mayo:
		Mayo &
		 &
		\begin{itemize}
			\item 
		\end{itemize} &
		\begin{itemize}
			\item 
		\end{itemize} &
		  &
		  &
		 - \\
		\hline
		% Plan de Junio:
		Junio &
		 &
		\begin{itemize}
			\item 
		\end{itemize} &
		\begin{itemize}
			\item 
		\end{itemize} &
		  &
		  &
		 - \\
		\hline
		% Plan de Julio:
		Julio &
		 &
		\begin{itemize}
			\item 
		\end{itemize} &
		\begin{itemize}
			\item 
		\end{itemize} &
		  &
		  &
		 - \\
		\hline
		% Plan de Agosto:
		Agosto &
		 &
		\begin{itemize}
			\item 
		\end{itemize} &
		\begin{itemize}
			\item 
		\end{itemize} &
		  &
		  &
		 - \\
		\hline
		% Plan de Septiembre:
		Septiembre &
		 &
		\begin{itemize}
			\item 
		\end{itemize} &
		\begin{itemize}
			\item 
		\end{itemize} &
		  &
		  &
		 - \\
		\hline
		% Plan de Octubre:
		Octubre &
		 &
		\begin{itemize}
			\item 
		\end{itemize} &
		\begin{itemize}
			\item 
		\end{itemize} &
		  &
		  &
		 - \\
		\hline
		% Plan de Noviembre:
		Noviembre &
		 &
		\begin{itemize}
			\item 
		\end{itemize} &
		\begin{itemize}
			\item 
		\end{itemize} &
		  &
		  &
		 - \\
		\hline

	\end{longtable}
	\pagebreak[4]
	% ----------------------------------------------------------------------------------------------------------------------------------------
	% Cabecera:
	\begin{tabularx}{\textwidth}{ >{\raggedright\arraybackslash}X >{\centering\arraybackslash}X >{\raggedleft\arraybackslash}X }
		\includegraphics[width=0.3\textwidth]{\CEGSA} &
		\textbf{PLAN ANUAL DE CLASES} &
		\includegraphics[width=0.3\textwidth]{\CETSA}
	\end{tabularx}
	% Identificación:
	\begin{tabularx}{\textwidth}{ >{\raggedright\arraybackslash}X >{\raggedright\arraybackslash}X >{\raggedright\arraybackslash}X }
		Docente: \profesor &
		Turno: - &
		Año: \currentyear \\
		Disciplina: \discipline &
		Grado/Curso: 2do Curso &
		 \\
	\end{tabularx}
	% Contenido:
	\centering
	\begin{longtable}{|m{\smallcellwidth}|p{\normalcellwidth}|p{\bigcellwidth}|p{\bigcellwidth}|p{\normalcellwidth}|p{\normalcellwidth}|p{\normalcellwidth}|}
	%\begin{tabularx}{\textwidth}{|r|r|r|r|r|r|r|}
		\hline
		\textbf{Mes} &
		\textbf{Contenido/Unidad Temática} &
		\textbf{Capacidades} &
		\textbf{Indicadores} &
		\textbf{Recursos Didácticos/Uso de TIC's} &
		\textbf{Instrumentos de Evaluación} &
		\textbf{Proyectos Disciplinarios} \\
		\hline
		\endhead
		% Plan de Febrero:
		Febrero &
		 &
		\begin{itemize}
			\item 
		\end{itemize} &
		\begin{itemize}
			\item 
		\end{itemize} &
		  &
		  &
		 - \\
		\hline
		% Plan de Marzo:
		Marzo &
		 &
		\begin{itemize}
			\item 
		\end{itemize} &
		\begin{itemize}
			\item 
		\end{itemize} &
		  &
		  &
		 - \\
		\hline
		% Plan de Abril:
		Abril &
		 &
		\begin{itemize}
			\item 
		\end{itemize} &
		\begin{itemize}
			\item 
		\end{itemize} &
		  &
		  &
		 - \\
		\hline
		% Plan de Mayo:
		Mayo &
		 &
		\begin{itemize}
			\item 
		\end{itemize} &
		\begin{itemize}
			\item 
		\end{itemize} &
		  &
		  &
		 - \\
		\hline
		% Plan de Junio:
		Junio &
		 &
		\begin{itemize}
			\item 
		\end{itemize} &
		\begin{itemize}
			\item 
		\end{itemize} &
		  &
		  &
		 - \\
		\hline
		% Plan de Julio:
		Julio &
		 &
		\begin{itemize}
			\item 
		\end{itemize} &
		\begin{itemize}
			\item 
		\end{itemize} &
		  &
		  &
		 - \\
		\hline
		% Plan de Agosto:
		Agosto &
		 &
		\begin{itemize}
			\item 
		\end{itemize} &
		\begin{itemize}
			\item 
		\end{itemize} &
		  &
		  &
		 - \\
		\hline
		% Plan de Septiembre:
		Septiembre &
		 &
		\begin{itemize}
			\item 
		\end{itemize} &
		\begin{itemize}
			\item 
		\end{itemize} &
		  &
		  &
		 - \\
		\hline
		% Plan de Octubre:
		Octubre &
		 &
		\begin{itemize}
			\item 
		\end{itemize} &
		\begin{itemize}
			\item 
		\end{itemize} &
		  &
		  &
		 - \\
		\hline
		% Plan de Noviembre:
		Noviembre &
		 &
		\begin{itemize}
			\item 
		\end{itemize} &
		\begin{itemize}
			\item 
		\end{itemize} &
		  &
		  &
		 - \\
		\hline

	\end{longtable}
	\pagebreak[4]
	% ----------------------------------------------------------------------------------------------------------------------------------------
	% Cabecera:
	\begin{tabularx}{\textwidth}{ >{\raggedright\arraybackslash}X >{\centering\arraybackslash}X >{\raggedleft\arraybackslash}X }
		\includegraphics[width=0.3\textwidth]{\CEGSA} &
		\textbf{PLAN ANUAL DE CLASES} &
		\includegraphics[width=0.3\textwidth]{\CETSA}
	\end{tabularx}
	% Identificación:
	\begin{tabularx}{\textwidth}{ >{\raggedright\arraybackslash}X >{\raggedright\arraybackslash}X >{\raggedright\arraybackslash}X }
		Docente: \profesor &
		Turno: - &
		Año: \currentyear \\
		Disciplina: \discipline &
		Grado/Curso: 3er Curso &
		 \\
	\end{tabularx}
	% Contenido:
	\centering
	\begin{longtable}{|m{\smallcellwidth}|p{\normalcellwidth}|p{\bigcellwidth}|p{\bigcellwidth}|p{\normalcellwidth}|p{\normalcellwidth}|p{\normalcellwidth}|}
	%\begin{tabularx}{\textwidth}{|r|r|r|r|r|r|r|}
		\hline
		\textbf{Mes} &
		\textbf{Contenido/Unidad Temática} &
		\textbf{Capacidades} &
		\textbf{Indicadores} &
		\textbf{Recursos Didácticos/Uso de TIC's} &
		\textbf{Instrumentos de Evaluación} &
		\textbf{Proyectos Disciplinarios} \\
		\hline
		\endhead
		% Plan de Febrero:
		Febrero &
		 &
		\begin{itemize}
			\item 
		\end{itemize} &
		\begin{itemize}
			\item 
		\end{itemize} &
		  &
		  &
		 - \\
		\hline
		% Plan de Marzo:
		Marzo &
		 &
		\begin{itemize}
			\item 
		\end{itemize} &
		\begin{itemize}
			\item 
		\end{itemize} &
		  &
		  &
		 - \\
		\hline
		% Plan de Abril:
		Abril &
		 &
		\begin{itemize}
			\item 
		\end{itemize} &
		\begin{itemize}
			\item 
		\end{itemize} &
		  &
		  &
		 - \\
		\hline
		% Plan de Mayo:
		Mayo &
		 &
		\begin{itemize}
			\item 
		\end{itemize} &
		\begin{itemize}
			\item 
		\end{itemize} &
		  &
		  &
		 - \\
		\hline
		% Plan de Junio:
		Junio &
		 &
		\begin{itemize}
			\item 
		\end{itemize} &
		\begin{itemize}
			\item 
		\end{itemize} &
		  &
		  &
		 - \\
		\hline
		% Plan de Julio:
		Julio &
		 &
		\begin{itemize}
			\item 
		\end{itemize} &
		\begin{itemize}
			\item 
		\end{itemize} &
		  &
		  &
		 - \\
		\hline
		% Plan de Agosto:
		Agosto &
		 &
		\begin{itemize}
			\item 
		\end{itemize} &
		\begin{itemize}
			\item 
		\end{itemize} &
		  &
		  &
		 - \\
		\hline
		% Plan de Septiembre:
		Septiembre &
		 &
		\begin{itemize}
			\item 
		\end{itemize} &
		\begin{itemize}
			\item 
		\end{itemize} &
		  &
		  &
		 - \\
		\hline
		% Plan de Octubre:
		Octubre &
		 &
		\begin{itemize}
			\item 
		\end{itemize} &
		\begin{itemize}
			\item 
		\end{itemize} &
		  &
		  &
		 - \\
		\hline
		% Plan de Noviembre:
		Noviembre &
		 &
		\begin{itemize}
			\item 
		\end{itemize} &
		\begin{itemize}
			\item 
		\end{itemize} &
		  &
		  &
		 - \\
		\hline

	\end{longtable}
	% ----------------------------------------------------------------------------------------------------------------------------------------
\end{document}
